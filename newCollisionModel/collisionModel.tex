\documentclass{article}

% Language setting
% Replace `english' with e.g. `spanish' to change the document language
%\usepackage[colorlinks=true, linkcolor=blue, urlcolor=blue, citecolor=blue]{hyperref}
\usepackage[english]{babel}
\usepackage{amsmath}
\usepackage{tikz}

% Set page size and margins
% Replace `letterpaper' with `a4paper' for UK/EU standard size
\usepackage[letterpaper,top=2cm,bottom=2cm,left=3cm,right=3cm,marginparwidth=1.75cm]{geometry}

% Useful packages
\usepackage{amsmath}
\usepackage{graphicx}
\usepackage[colorlinks=true, allcolors=blue]{hyperref}

\title{\textbf{New Collision model}}
\author{Jian Eduardo Cardenas Cabezas}

\begin{document}
\maketitle

\section{Modification of the classical BGK operator}
The classical BGK is a collision operator really easy to implement. It could be interpreted as the relaxation time needed to accomplish the equilibrium. The problem thai it has is that it lacks from stability point of view. In order to improve stability
in the classical control theory it could be interpreted as the inversed of the P part in the PID control. Suddenly, according to control theory, if we use a big value of proportionality 
it could be unstable. Then, in order to improve stability the idea is to add the derivative part that is going to reduce the oscilations. The purpose of this document is to  
show that adding the derivative part could increase stability without changing the viscosity.
The new collision operator has the shape of:

\begin{equation}
    \Omega(f_i) = -\frac{\sigma}{\tau}(f_i-f^{eq}) - \frac{(1-\sigma)}{\tau}\frac{d(f_i-f^{eq})}{dt}    
\end{equation}

\section{Discretization of the LBE using the new collision operator}

The lattice Boltzmann equation is: 

\begin{equation}
    \frac{\partial f_{i}}{\partial t} +c_{i\alpha }\frac{\partial f_{i}}{\partial x_{\alpha }} =\Omega _{i} =\frac{\mathrm{d} f_{i}}{\mathrm{d} t}
\end{equation}

This expression is known as a . in order to compute a calculation of properties some interesting relationships are needed:

\section{Toy model to verify stability properties}

\section{Chapman Enskog expansion of the new collision model}

\end{document}