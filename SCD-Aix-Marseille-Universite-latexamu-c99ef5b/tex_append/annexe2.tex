\chapter{Unified Hybrid Recursive Regularised LBM}
\label{uHRR-LBM}

The theoretical foundation of the method is briefly reviewed. The Lattice Boltzmann equation (LBE), which governs the time evolution of the particle distribution function, is formulated as:

\begin{equation}
    \frac{\partial f_i}{\partial t} + c_{i\alpha} \frac{\partial f_i}{\partial x_{\alpha}} = -\frac{1}{\tau}(f_i - f_i^{\mathrm{eq}}) + F_i,
\end{equation}
where $f_i$ denotes the distribution function along the $i$-th discrete velocity direction, and $f_i^{\mathrm{eq}}$ is its corresponding equilibrium value. Here, $c_{i\alpha}$ is the $\alpha$-component of the lattice velocity vector $\mathbf{c}_i$, and $\tau$ is the relaxation time. The term $F_i$ denotes an external force applied in direction $i$.

In this paper, the equilibrium function given by the uHRR with \(\kappa\) and \(\xi\) set to zero is used to obtain a method similar to the pressure-based approach proposed in~\cite{farag2020pressure}:

\begin{equation}
    f_i^{eq} = \omega_i\left( \rho + \frac{\omega_i-\delta_{0i}}{\omega_i}(\frac{P}{c_s^2}-\rho) + \frac{\mathcal{H}_{i\alpha}^{(1)}}{c_s^2} \rho u_\alpha + \frac{\mathcal{H}_{i\alpha\beta}^{(2)}}{2c_s^4}[\rho u_\alpha u_\beta] + \frac{\mathcal{H}_{i\alpha\beta\gamma}^{(3)}}{6c_s^6}[\rho u_\alpha u_\beta u_\gamma] \right)
\end{equation}

The details and signification of each component of the equation can be found in ~\cite{farag2021unified}. This method adopts the D3Q19 velocity discretization, where the raw moments of the equilibrium distribution are defined as:

\begin{equation}
    \sum_i f_i^{\mathrm{eq}} = \sum_i f_i = \rho,
\end{equation}

\begin{equation}
    \sum_i c_{i\alpha} f_i^{\mathrm{eq}} = \sum_i c_{i\alpha} f_i = \rho u_{\alpha},
\end{equation}

\begin{equation}
    \sum_i c_{i\alpha} c_{i\beta} f_i^{\mathrm{eq}} = \rho u_{\alpha} u_{\beta} + P \delta_{\alpha\beta}.
\end{equation}

These raw moments directly yield the macroscopic quantities of interest at each time step. A more rigorous derivation of the governing hydrodynamic equations can be achieved using the Chapman-Enskog or Taylor expansion techniques, as demonstrated in~\cite{kruger2017lattice}.

Unlike the classical athermal LBM, where the pressure is defined as \( P = \rho c_s^2 \), a parameter \( \theta \) is introduced to incorporate thermal effects. In uHRR formulation, \( \theta \) is defined as:

\begin{equation}
    \theta = \frac{P}{\rho c_s^2},
\end{equation}

where \( P \) is the real pressure defined by an Equation of State (EoS) and \( \rho c_s^2 \) represents the athermal pressure. This formulation decouples \( \theta \) from the temperature, enabling the method to represent non-ideal and supercritical thermodynamic behavior independently of the chosen EoS.



To achieve second-order accuracy in time, the Crank--Nicolson scheme is applied to discretize the LBE. To maintain an explicit formulation, a change of variables is introduced as:

\begin{equation}
    \bar{f}_i = f_i - \frac{\Delta t}{2\tau}(f_i^{\mathrm{eq}} - f_i) - \frac{\Delta t}{2} F_i,
\end{equation}

\begin{equation}
    \bar{\tau} = \tau + \frac{\Delta t}{2}.
\end{equation}

where $\Delta t$ is the time-step. Using this transformation, the algorithm is split into two steps. The collision step, a local operation, is expressed as:

\begin{equation}
    \bar{f}_i^{\mathrm{col}} = f_i^{\mathrm{eq}} + \left(1 - \frac{\Delta t}{\bar{\tau}}\right) \bar{f}_i^{\mathrm{neq}} + \frac{\Delta t}{2} F_i,
\end{equation}

followed by the streaming step:

\begin{equation}
    \bar{f}_i(t + \Delta t, \mathbf{x}) = \bar{f}_i^{\mathrm{col}}(t, \mathbf{x} - \mathbf{c}_i \Delta t).
\end{equation}

After this variable change, the computation of macroscopic variables is made using:
\begin{equation}
    \rho = \sum_i \Bar{f}_i
\end{equation}

\begin{equation}
    \rho u_\alpha= \sum_i c_{i\alpha}\Bar{f}_i + \frac{\Delta t}{2}\sum_i c_{i\alpha}F_i
\end{equation}

In this formulation, the only unresolved component is the non-equilibrium part $\bar{f}_i^{\mathrm{neq}}$, which will be detailed in the next subsection.


\subsection{Non-Equilibrium reconstruction}

In this study, the hybrid recursive regularization approach proposed in~\cite{farag2021unified} is adopted, which combines the recursive regularization technique with traceless non-equilibrium moment reconstruction~\cite{farag2020pressure}. This method has been shown to serve as an effective supplementary regularization strategy, particularly for improving the accuracy of the second-order non-equilibrium moment $\Pi_{\gamma\gamma}$~\cite{wissocq2022hydrodynamic}:

\begin{equation}
\begin{aligned}
\Pi_{\alpha\beta}^{\bar{f}^{\text{neq}},(2)}(x,t) =\ 
&\sigma \sum_i \left( c_{i\alpha} c_{i\beta} - \frac{\delta_{\alpha\beta}}{3} c_{i\gamma} c_{i\gamma} \right) 
\left[ \bar{f}_i(x,t) - f_i^{\text{eq}}(x,t) + \frac{\Delta t}{2} F_i(x,t - \Delta t) \right] \\
&- (1 - \sigma) \rho c_s^2 \bar{\tau} \left( 
\frac{\partial u_\alpha}{\partial x_\beta} + \frac{\partial u_\beta}{\partial x_\alpha} 
- \frac{2 \delta_{\alpha\beta}}{3} \frac{\partial u_\gamma}{\partial x_\gamma} 
\right)(x,t)
\end{aligned}
\end{equation}

where $\sigma$ is a free weighting parameter introduced in~\cite{jacob2018new}. Note that the force term $F_i$ is evaluated at time $t - \Delta t$ for stability reasons as mentioned in ~\cite{farag2021unified}.

The recursive regularization procedure defines the non-equilibrium distribution as follows:

\begin{equation}
    \Pi_{\alpha\beta\gamma}^{\Bar{f}^{neq}}(x,t) = \left[ u_\alpha \Pi_{\beta\gamma}^{\Bar{f}^{neq},(2)} + u_\beta \Pi_{\gamma\alpha}^{\Bar{f}^{neq},(2)} + u_\gamma \Pi_{\alpha\beta}^{\Bar{f}^{neq},(2)}\right](x,t)
\end{equation}

Then, the recursive D3Q19r (~\ref{appendix:D3Q19r}) dictates that the non equilibrium, is defined as:

\begin{equation}
    \Bar{f}^{neq}_i = \omega_i\left( \frac{\mathcal{H}_{i\alpha\beta}^{(2)}}{2c_s^4} \Pi_{\alpha\beta}^{\Bar{f}^{neq},(2)} + \frac{\mathcal{H}_{i\alpha\beta\gamma}^{(3r)}}{6c_s^6} \Pi_{\alpha\beta}^{\Bar{f}^{neq},(2)}\right)
\end{equation}
