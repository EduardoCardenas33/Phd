\chapter{Fluid simulation}
\chaptertoc{}

In fluid simulation, there are different approaches: the microscopic, the
mesoscopic, and the macroscopic approach. In the microscopic approach, all the
interactions between molecules are modeled. While this method provides a highly
detailed representation of the fluid's behavior, it is extremely time-consuming
due to the complexity of calculating every molecular interaction.

The mesoscopic approach, on the other hand, is based on kinetic theory. In this
approach, a group of molecules is treated as a single entity. This method relies
on statistical mechanics to describe the distribution of particles in a fluid
and how this distribution evolves over time. It provides a balance between
computational efficiency and accuracy by focusing on the collective behavior of
particles rather than individual interactions ~\cite{kruger2017lattice}.

The macroscopic approach is the classical method, where the fluid is treated by
considering only its macroscopic properties, such as velocity, pressure, and
temperature. This method typically involves solving the Navier-Stokes (NS)
equations, which describe the motion of fluid substances.

In this manuscript, the mesoscopic approach is utilized, specifically through
 the Lattice Boltzmann Method (LBM). The LBM has proven its efficiency in
 simulating fluid dynamics by significantly improving computational speed
 compared to classical NS solvers ~\cite{boivin2021benchmarking}.

To achieve this, some fundamental concepts from kinetic theory are introduced,
followed by an overview of the classical Lattice Boltzmann Method. Subsequently,
the method used for the simulation of supercritical CO\textsubscript{2} is
presented.

\section{Fundamentals on kinetic theory}

\section{Fundamentals on Lattice Boltzmann Method}

The Lattice Boltzmann Method (LBM) was developed in the late 1980s and has been
used for Computational Fluid Dynamics (CFD) since then. Specifically, the
foundational work on LBM can be traced back to the work of Frisch, Hasslacher,
and Pomeau in 1985 ~\cite{frisch1985lattice}, who introduced the Lattice Gas
Automaton (LGA), which is the precursor to LBM. The LBM, as it is known today,
evolved from these early developments in the late 1980s and early 1990s. This
first method was only used for weakly compressible flows, where the temperature
is constant and a perfect gas equation is used to close the system and being
equivalent to a Navier-Stokes formulation. The method has evelouated from that
basic formulation and nowadays permits the use of difficult equations of state
and the use for High compressible flows. This section provides the main concepts
and expressions used in the classical and more simplest LBM. For doing that, a
brief introduction of the Boltzmann equation is made. 

\subsection{Boltzmann equation}
The Boltzmann equation, formulated by Ludwig Boltzmann in 1872, is a fundamental
equation in statistical mechanics that describes the statistical behavior of a
thermodynamic system out of equilibrium. It is used to study the dynamics of a
gas at the microscopic level by considering the distribution function
f($x$,$\xi$,$t$), which represents the number of particles at a given position
$x$, with a given velocity $\xi$, at time $t$.

In order to see the time evolution, the total derivative of the distribution
function can be expressed as:

\begin{equation}
	\frac{\mathrm{d}f}{\mathrm{d}t} = 
	\left(\frac{\partial f}{\partial t}\right)\frac{\mathrm{d}t}{\mathrm{d}t}
	+\left(\frac{\partial f}{\partial x_{\beta}}\right)\frac{\mathrm{d}x_{\beta}}{\mathrm{d}t}
	+\left(\frac{\partial f}{\partial \xi_{\beta}}\right)\frac{\mathrm{d}\xi_{\beta}}{\mathrm{d}t}
\end{equation}

Looking each term, we can indentify and define $\frac{\mathrm{d}f}{\mathrm{d}t}
= \Omega(f)$ as the collision operator, $\frac{\mathrm{d}t}{\mathrm{d}t}=1$, the
particle velocity as $\frac{\mathrm{d}x_{\beta}}{\mathrm{d}t} = \xi_{\beta}$ and
$\frac{\mathrm{d}\xi_{\beta}}{\mathrm{d}t} = \frac{F_{\beta}}{\rho}$.

Leading to the final expresion for the Boltzmann equation.

\begin{equation}
	\Omega(f) = \frac{\partial f}{\partial t} 
	+ \xi_{\beta}\frac{\partial f}{\partial x_{\beta}}
	+ \frac{F_{\beta}}{\rho}\frac{\partial f}{\partial \xi_{\beta}}
	\label{Boltzmann equation}
\end{equation}

This expression can be seen as an advection equation for the distribution
function, where the collision operator acts as a source term.

In the first equation developed by Ludwig Boltzmann, the formulation for the
collision operator involved complex integrals and cumbersome mathematical
operations, making the calculations difficult to achieve. In 1954, P. L.
Bhatnagar, E. P. Gross, and M. Krook introduced the BGK collision operator in
their seminal paper titled "A Model for Collision Processes in Gases. I. Small
Amplitude Processes in Charged and Neutral One-Component Systems," published in
Physical Review. The BGK collision operator simplifies these calculations by
incorporating only a single relaxation time to the equilibrium distribution
function ~\cite{bhatnagar1954model}. This equailibrium distribution function is
expressed in a continous form thanks to the Maxwellian distribution function:

\begin{equation}
	f^{eq} = \frac{\rho}{(2\pi RT)^{D/2}}exp\left(-\frac{|\xi - u|^2}{2RT}\right)
\end{equation}


\subsection{Distribution Function Projection onto Hermite Series Basis}

The continuous Boltzmann equation presented in the previous section must now be
discretized to enable numerical solutions. This involves discretization not only
in time and space but also in velocity space, denoted as $\xi$. The distribution
function $f$ in Eq. ~\ref{Boltzmann equation} must be projected onto an
appropriate functional space to serve as the basis for discretization and
expansion procedures.

In 1998, Shan and He ~\cite{shan1998discretization} demonstrated that the
velocity space can be discretized following the framework established by Grad
and his 13-moment equation ~\cite{grad1949note}. Grad's approach extended the
Hermite polynomial distribution function, elaborating on its mathematical
properties and utility for fluid models. This methodology was further developed
by He and Luo in 1997 ~\cite{he1997theory} and refined by Shan and He in 1998
~\cite{shan1998discretization}. Later, in 2006, Shan and collaborators
~\cite{shan2006kinetic} constructed a comprehensive discretization model using
Gauss-Hermite quadrature on a grid.

Importantly, any continuous function, including the equilibrium distribution
function, can be expressed as a series expansion in terms of Hermite
polynomials: 

\begin{equation}
	f^{eq}_{(x,\xi,t)} = \omega(\xi)\sum_{n=0}^{\infty}\frac{1}{n!(rT_0)^n}a^{(n),eq}_{(x,t)}:\mathcal{H}^n_{(\xi)}
	\label{equilibrium distribution function}
\end{equation}

Here, $\omega$, $\mathcal{H}^(n)$, and $a^{(n),eq}$ represent, respectively, the
weight function, the Hermite polynomials, and the Hermite expansion
coefficients. Their definitions are given as follows:

\begin{equation}
	\omega_{(\xi)} = \frac{1}{(2\pi rT_0^{D/2})}exp\left(-\frac{|\xi|^2}{2rT_0}\right)
\end{equation}

\begin{equation}
	\mathcal{H}^{(n)}_{(\xi)} = \frac{(-rT_0)^n}{\omega_(\xi)}\nabla_{\xi}^n\omega_{\xi}
\end{equation}

\begin{equation}
	a^{(n),eq}_{(x,t)} = \int f^{eq}_{(x,\xi,t)}\mathcal{H}^{(n)}_{(\xi)}\mathrm{d}\xi
	\label{a_eq equation}
\end{equation}

Here, the weight (or generative) function $\omega$ of the Hermite polynomials is
normalized such that $\int\omega_{(\xi)}\mathrm{d}\xi=1$. Additionally, the
operator $\nabla_\xi^n$ denotes the n-th order gradient tensor, obtained through
n successive differentiations with respect to $\xi$. Finally, the inner product
is defined in the space of square-integrable functions, relative to the
orthogonal Hermite polynomial basis, as:

\begin{equation}
	<a|b> = \int\omega_{(\xi)}a_{(\xi)}b_{(\xi)}\mathrm{d}\xi
\end{equation}

The first Hermite polynomials are:

\begin{equation}
	\begin{aligned}
		\mathcal{H}^{(0)} &= 1 \\
		\mathcal{H}_{\alpha}^{(1)} &= \xi_{\alpha} \\
		\mathcal{H}_{\alpha\beta}^{(2)} &= \xi_{\alpha}\xi_{\beta} - rT_0\delta_{\alpha\beta} \\
		\mathcal{H}_{\alpha\beta\gamma}^{(3)} &= \xi_{\alpha}\xi_{\beta}\xi_{\gamma} 
		- rT_0\left(\delta_{\alpha\beta}\xi_{\gamma} + \delta_{\alpha\gamma}\xi_{\beta} + \delta_{\beta\gamma}\xi_{\alpha}\right)\\
		\mathcal{H}_{\alpha\beta\gamma\delta}^{(4)} &= \xi_{\alpha}\xi_{\beta}\xi_{\gamma}\xi_{\delta} \\
    	&\quad - rT_0 \big(
    	    \delta_{\alpha\beta}\xi_{\gamma}\xi_{\delta} 
    	    + \delta_{\alpha\gamma}\xi_{\beta}\xi_{\delta} 
    	    + \delta_{\alpha\delta}\xi_{\beta}\xi_{\gamma} \\
    	&\quad 
    	    + \delta_{\beta\gamma}\xi_{\alpha}\xi_{\delta} 
    	    + \delta_{\beta\delta}\xi_{\alpha}\xi_{\gamma} 
    	    + \delta_{\gamma\delta}\xi_{\alpha}\xi_{\beta} 
    	\big) \\
    	&\quad + (rT_0)^2 \big(
    	    \delta_{\alpha\beta}\delta_{\gamma\delta} 
    	    + \delta_{\alpha\gamma}\delta_{\beta\delta} 
    	    + \delta_{\alpha\delta}\delta_{\beta\gamma}
    	\big)	
	\end{aligned}
\end{equation}	

From Equation ~\ref{a_eq equation}, the first four equilibrium moments can be
derived as follows:

\begin{equation}
	\begin{aligned}
		a^{(0,eq)} &= \rho\\
		a^{(1,eq)} &= \rho u\\
		a^{(2,eq)} &= \rho\left[u^2 + rT_0(\theta -1)\delta\right]\\
		a^{(3,eq)} &= \rho\left[u^3 + rT_0(\theta -1)u\delta\right]\\
		a^{(4,eq)} &= \rho\left[u^4 + rT_0(\theta -1)u^2\delta + (rT_0)^2(\theta - 1)\delta^2\right]
	\end{aligned}
\end{equation}

Here, $\theta=\frac{T}{T_0}$. It is now evident why Hermite polynomials are
particularly well-suited for projecting the equilibrium function. In this basis,
the expansion coefficients naturally align with the hydrodynamic (i.e.,
low-order) moments of the equilibrium function.

\subsection{Lattice Boltzmann equation discretization}

The lattice Boltzmann (LB) equation presented earlier must now be discretized.
First, the velocity discretization is introduced. This is followed by a
discussion of spatial and temporal discretization methods. Finally, the
limitations of these simplified models are outlined, providing context and
motivation for the developments discussed in the subsequent sections.

The continuous velocity space $\xi_i$ must be discretized into a finite set of
velocities $\xi_i$ while ensuring that the conservation laws for the discrete
moments $f_i^{eq}$ are preserved. This is achieved by applying the Gauss-Hermite
quadrature rule:

\begin{equation}
	\int \mathcal{H}^{(M)}f^{(N),eq}\mathrm{d}\xi = \sum_i \mathcal{H}_i^{(M)}f_i^{(N),eq}
	\label{Gauss-Hermite quadrature}
\end{equation}

Here, $M$ represents the order of the Hermite polynomials, and $N$ denotes the
truncation order of Equation ~\ref{equilibrium distribution function}, which
will be discussed shortly. Additionally, the following shorthand notations are
introduced:

\begin{equation}
	\mathcal{H}_i^{(M)} = \mathcal{H}^{(M)}(\xi) 
\end{equation}

\begin{equation}
	f_i^{(N),eq} = \frac{\omega_i}{\omega(\xi)}f^{(N),eq}(\xi) 
\end{equation}

The weights $\omega_i$ in the Gaussian quadrature represent the Gaussian
quadrature weights. Applying the Gauss-Hermite quadrature from Eq.
~\ref{Gauss-Hermite quadrature} to a selected lattice set enables the
calculation of moments and coefficients for the Hermite series expansion. This
quadrature is performed using a truncated Hermite expansion of order N, which
introduces various errors depending on the truncation order. As explained in
~\cite{farag2022modelisation}, due to the low quadrature order of standard
lattices, a third-order equilibrium function is required to correctly capture
the first two moments.

    Second-order truncation introduces significant errors in the energy equation
    but is suitable for athermal low-Mach-number cases. Third-order truncation
    recovers the second-order moment but fails to correctly model the energy
    equation. Fourth-order truncation ensures accurate modeling of the energy
    equation, enabling the simulation of thermo-hydrodynamic flows.

It is worth noting that the truncation accuracy strongly depends on the chosen
velocity set. A fully isotropic fourth-order truncation is only achievable with
very large velocity sets, which significantly increase computational costs.
Standard velocity sets, commonly referred to as nearest-neighbor lattices,
are typically used for low-Mach, athermal applications. These standard lattices
only support truncation up to N=2, leading to the well-known isotropy problem
~\cite{kruger2017lattice}.

In this work, a standard lattice is employed to reduce computational cost,
complemented by appropriate corrections to ensure consistent energy
conservation. This approach aligns with prior contributions to combustion
modeling by Tayyab, Bhairapurada, Taileb, Boivin, and Zhao
~\cite{tayyab2020hybrid, tayyab2021lattice, zhao2023lattice, taileb2022lattice},
as well as to compressible flow simulations by Coratger, Farag, and Wissocq
~\cite{coratger2021large, wissocq2022restoring, zhao2020toward}.

The design of velocity sets involves rescaling $\xi_i$ to obtain on-grid lattice
velocities $c_i$. These velocity sets are conventionally denoted as $DdQq$,
where d is the number of spatial dimensions, and q is the number of discrete
velocities in the lattice. Three components are essential to fully define a
lattice set: the discrete velocities $c_i$, their associated weights $\omega_i$,
and the derived lattice sound speed $c_s$.

Furthermore, for a lattice to be suitable for any lattice Boltzmann (LB) method
and to serve as an accurate Navier-Stokes solver, the lattice weights must be
positive and isotropic up to the fifth order. This requirement translates into
satisfying the following identities:

\begin{equation}
	\begin{aligned}
		\sum_i \omega_i &= 1, \\
		\sum_i \omega_i c_{i\alpha} &= 0, \\
		\sum_i \omega_i c_{i\alpha} c_{i\beta} &= c_s^2 \delta_{\alpha\beta}, \\
		\sum_i \omega_i c_{i\alpha} c_{i\beta} c_{i\gamma} &= 0, \\
		\sum_i \omega_i c_{i\alpha} c_{i\beta} c_{i\gamma} c_{i\delta} &= c_s^4 (\delta_{\alpha\beta}\delta_{\gamma\delta} + \delta_{\alpha\gamma}\delta_{\beta\delta} + \delta_{\alpha\delta}\delta_{\beta\gamma}), \\
		\sum_i \omega_i c_{i\alpha} c_{i\beta} c_{i\gamma} c_{i\delta} c_{i\epsilon} &= 0.
	\end{aligned}
\end{equation}
	

The distribution function f is discretized similarly to $f^{eq}$ 

\begin{equation}
	f_i(x,t) = \frac{w_i}{w(c_i)}f(x, c_i,t)
\end{equation}

Finally, the only expression that missed is the collision operator.

The next subsection is dedicated to show a little bit about the BGK
collision operator and how it was constructed.

\subsection{BGK Collision Operator}

The BGK collision operator can be seen as a linearization of the complicated
collision operator proposed by Boltzmann. Like any other collision operator, it
must respect the conservation of mass, momentum, and energy. To illustrate how
its form is constructed, a Taylor expansion of the collision operator around the
equilibrium point is developed.

\begin{equation}
	\Omega(f) = \Omega|_{f^{eq}} 
	+ \Delta f \frac{\partial \Omega}{\partial f} \bigg|_{f^{eq}} 
	+ \mathrm{O}(\Delta f^2)
\end{equation}

As explained before, the collision operator evaluated at the equilibrium should
be equal to zero. Thus, the linearized expression for the collision operator is
represented by the following equation:

\begin{equation}
	\Omega(f) \approx 
	(f - f^{eq}) \frac{\partial \Omega}{\partial f} \bigg|_{f^{eq}} 
\end{equation}

Typically, this first derivative evaluated at the equilibrium point is
considered as the relaxation parameter, which indicates that over a long time,
equilibrium is achieved.

\begin{equation}
	\frac{\partial \Omega}{\partial f} \bigg|_{f^{eq}} = -\frac{1}{\tau}	
\end{equation}

Finally, the Boltzmann equation using the BGK collision operator is written as:

\begin{equation}
	\frac{\partial f}{\partial t} 
	+ \xi_{\beta} \frac{\partial f}{\partial x_{\beta}}
	+ \frac{F_{\beta}}{\rho} \frac{\partial f}{\partial \xi_{\beta}} = -\frac{1}{\tau}(f - f^{eq})
\end{equation}



\subsection{Numerical Discretization}
In this section , a brief explanation about how to discrretise the Lattice
Boltzmann equation is made.

In order to make the discretization, the method of Characteristics and the Crank
Nicholson was used.

After using the Hermite polynomials the discrete LB equation is:

\begin{equation}
    \frac{\partial f_i}{\partial t} + c_{i\alpha}\frac{\partial f_i}{\partial x_\alpha} = \Omega_i +F_i 
\end{equation}

Applying the characteristic method and Crank - Nicholson pour la discretisation we have: 

\begin{equation}
    f_{i_{(t+\Delta t, x_i +c_i \Delta t)}} = f_{i_{(t,x)}} + \int_{t}^{t+\Delta t}(\Omega_i + F_i)d\xi
\end{equation}

\begin{equation*}
    f_{i_{(t+\Delta t, x_i +c_i \Delta t)}} = f_{i_{(t,x)}} + \frac{\Delta t}{2}((\Omega_i +F_i)_{(t+\Delta t, x_i +c_i \Delta t)} + (\Omega_i +F_i)_{(t, x_i)})
\end{equation*}

This equation is implicit. In order to have some explicit scheme, we can use a
change of variable $\Bar{f_i} = f_i - \frac{\Delta t}{2}(\Omega_i + F_i) $. Note
that this substitution is valable for all the time of evaluation.

After making this change of variable, we have:
\begin{equation}
    \Bar{f_i}_{(t+\Delta t, x_i +c_i \Delta t)} = \Bar{f_i}_{(t,x)} + \Delta t(\Omega_i +F_i)_{(t, x_i)}
    \label{explicit equivalent equation}
\end{equation}

Even if this equation now is implicit, more algebraic manipulations can be made
in order to recover fastly the viscous stress tensor from the LBM (We are going
to see its importance when the Hybrid recursive regularised method is used to
have a more stable solver). To make that, two new variables variables
$\bar{\tau}$ and $\bar{f^{neq}}$ should be defined:

\begin{equation}
	\bar{\tau} = \tau + \frac{\Delta t}{2}
\end{equation}

\begin{equation}
	\bar{f}^{neq} = \bar{f} - f^{eq} - \frac{\Delta t}{2}F_i = f-f^{eq} - \frac{\Delta t}{2}\Omega_i  
\end{equation}

The last equation can be rewrite as:

\begin{equation}
    \Bar{f_i}_{(t+\Delta t, x_i +c_i \Delta t)} = \Bar{f_i}_{(t,x)} + \Delta t(-\frac{f_i - f_i^{eq}}{\tau} +F_i)_{(t, x_i)}
\end{equation}

\section{Unified Lattice Boltzmann method}
The basics of the Lattice Boltzmann Method (LBM) have been presented. After
performing the Chapman-Enskog expansion, it becomes apparent that a primary
limitation of the standard LBM is its reliance on the perfect gas equation of
state ($P$$=$$\rho$$c_s^2$).

However, to accurately simulate supercritical fluids, a more sophisticated
equation of state is required, as explained in the chapter on thermodynamics.
For capturing the properties of supercritical \ce{CO2}, an equation
such as the Peng-Robinson equation of state is necessary.

In the LBM community, there are two main approaches to address this limitation:
improved density-based models and pressure-based models. In the improved
density-based models, the second moment is modified to include a free parameter
that allows for the adjustment of pressure as needed. In the pressure-based
models, the zeroth moment is altered to directly use pressure as a variable.

To unify these two models and address the limitation related to the equation of
state, Farag et al. proposed the Unified Hybrid Recursive Regularized (UHRR)
model ~\cite{farag2021unified}. This model allows the use of any equation of
state, overcoming the constraints of the traditional LBM. Depending on the
parameters $\kappa$ or $\zeta$, the model can represent either the
pressure-based or density-based algorithm, for more details see
~\cite{farag2021unified}.

This section aims to present the essential differentiating features and the
algorithm used in the Unified Hybrid Recursive Regularized model. The main
difference from the classical LBM lies in the redefinition of the equilibrium
distribution function.


\begin{align}
	\begin{split}
		f_{i}^{eq} = \omega_{i} & \left\{ \rho + \frac{\omega_{i} - \delta_{0i}}{\omega_{i}} \rho \left[ \theta - 1 \right] (1 - \zeta) + \frac{H_{i\alpha}^{(1)}}{c_{s}^{2}} \rho u_{\alpha} + \frac{H_{i\alpha\beta}^{(2)}}{2c_{s}^{4}} \left[ \rho u_{\alpha} u_{\beta} + \zeta \delta_{\alpha\beta} \rho c_{s}^{2} (\theta - 1) \right] \right.\\ 
		+ &  \left.  \frac{H_{i\alpha\beta\gamma}^{(3)}}{6c_{s}^{6}} \left[ \rho u_{\alpha} u_{\beta} u_{\gamma} - \kappa \rho c_{s}^{2} (u_{\alpha} \delta_{\beta\gamma} + u_{\beta} \delta_{\gamma\alpha} + u_{\gamma} \delta_{\alpha\beta}) \right] \right\}
	\end{split}
\end{align}

In this work, the equilibrium function given by the uHRR with $\kappa$ and
$\zeta$ set to zero is used to obtain a method similar to the pressure-based
approach as mentioned in ~\cite{farag2021unified}.

This method adopts the D3Q19 velocity discretization, where the raw moments of
the equilibrium distribution are defined as:

\begin{equation}
    \sum_i f_i^{\mathrm{eq}} = \sum_i f_i = \rho,
\end{equation}

\begin{equation}
    \sum_i c_{i\alpha} f_i^{\mathrm{eq}} = \sum_i c_{i\alpha} f_i = \rho u_{\alpha},
\end{equation}

\begin{equation}
    \sum_i c_{i\alpha} c_{i\beta} f_i^{\mathrm{eq}} = \rho u_{\alpha} u_{\beta} + P \delta_{\alpha\beta}.
\end{equation}

These raw moments directly yield the macroscopic quantities of interest at each time step. A more rigorous derivation of the governing hydrodynamic equations can be achieved using the Chapman-Enskog or Taylor expansion techniques, as demonstrated in~\cite{kruger2017lattice}.

Unlike the classical athermal LBM, where the pressure is defined as \( P = \rho c_s^2 \), a parameter \( \theta \) is introduced to incorporate thermal effects. In uHRR formulation, \( \theta \) is defined as:

\begin{equation}
    \theta = \frac{P}{\rho c_s^2},
\end{equation}

where \( P \) is the real pressure defined by an Equation of State (EoS) and \( \rho c_s^2 \) represents the athermal pressure. This formulation decouples \( \theta \) from the temperature, enabling the method to represent non-ideal and supercritical thermodynamic behavior independently of the chosen EoS.



To achieve second-order accuracy in time, the Crank--Nicolson scheme is applied to discretize the LBE. To maintain an explicit formulation, a change of variables is introduced as:

\begin{equation}
    \bar{f}_i = f_i - \frac{\Delta t}{2\tau}(f_i^{\mathrm{eq}} - f_i) - \frac{\Delta t}{2} F_i,
\end{equation}

\begin{equation}
    \bar{\tau} = \tau + \frac{\Delta t}{2}.
\end{equation}

where $\Delta t$ is the time-step. Using this transformation, the algorithm is split into two steps. The collision step, a local operation, is expressed as:

\begin{equation}
    \bar{f}_i^{\mathrm{col}} = f_i^{\mathrm{eq}} + \left(1 - \frac{\Delta t}{\bar{\tau}}\right) \bar{f}_i^{\mathrm{neq}} + \frac{\Delta t}{2} F_i,
\end{equation}

followed by the streaming step:

\begin{equation}
    \bar{f}_i(t + \Delta t, \mathbf{x}) = \bar{f}_i^{\mathrm{col}}(t, \mathbf{x} - \mathbf{c}_i \Delta t).
\end{equation}

After this variable change, the computation of macroscopic variables is made using:
\begin{equation}
    \rho = \sum_i \Bar{f}_i
\end{equation}

\begin{equation}
    \rho u_\alpha= \sum_i c_{i\alpha}\Bar{f}_i + \frac{\Delta t}{2}\sum_i c_{i\alpha}F_i
\end{equation}

In this formulation, the only unresolved component is the non-equilibrium part $\bar{f}_i^{\mathrm{neq}}$, which will be detailed in the next subsection.


\subsection{Non-Equilibrium reconstruction}

In this study, the hybrid recursive regularization approach proposed in ~\cite{farag2021unified} is adopted, which combines the recursive regularization technique with traceless non-equilibrium moment reconstruction~\cite{farag2020pressure}. This method has been shown to serve as an effective supplementary regularization strategy, particularly for improving the accuracy of the second-order non-equilibrium moment $\Pi_{\gamma\gamma}$~\cite{wissocq2022hydrodynamic}:

\begin{equation}
\begin{aligned}
\Pi_{\alpha\beta}^{\bar{f}^{\text{neq}},(2)}(x,t) =\ 
&\sigma \sum_i \left( c_{i\alpha} c_{i\beta} - \frac{\delta_{\alpha\beta}}{3} c_{i\gamma} c_{i\gamma} \right) 
\left[ \bar{f}_i(x,t) - f_i^{\text{eq}}(x,t) + \frac{\Delta t}{2} F_i(x,t - \Delta t) \right] \\
&- (1 - \sigma) \rho c_s^2 \bar{\tau} \left( 
\frac{\partial u_\alpha}{\partial x_\beta} + \frac{\partial u_\beta}{\partial x_\alpha} 
- \frac{2 \delta_{\alpha\beta}}{3} \frac{\partial u_\gamma}{\partial x_\gamma} 
\right)(x,t)
\end{aligned}
\end{equation}

where $\sigma$ is a free weighting parameter introduced in~\cite{jacob2018new}. Note that the force term $F_i$ is evaluated at time $t - \Delta t$ for stability reasons as mentioned in ~\cite{farag2021unified}.

The recursive regularization procedure defines the non-equilibrium distribution as follows:

\begin{equation}
    \Pi_{\alpha\beta\gamma}^{\Bar{f}^{neq}}(x,t) = \left[ u_\alpha \Pi_{\beta\gamma}^{\Bar{f}^{neq},(2)} + u_\beta \Pi_{\gamma\alpha}^{\Bar{f}^{neq},(2)} + u_\gamma \Pi_{\alpha\beta}^{\Bar{f}^{neq},(2)}\right](x,t)
\end{equation}

Then, the recursive D3Q19r (~\ref{appendix:D3Q19r}) dictates that the non equilibrium, is defined as:

\begin{equation}
    \Bar{f}^{neq}_i = \omega_i\left( \frac{\mathcal{H}_{i\alpha\beta}^{(2)}}{2c_s^4} \Pi_{\alpha\beta}^{\Bar{f}^{neq},(2)} + \frac{\mathcal{H}_{i\alpha\beta\gamma}^{(3r)}}{6c_s^6} \Pi_{\alpha\beta}^{\Bar{f}^{neq},(2)}\right)
\end{equation}

\subsection{Standard LBM limitations}

Using first-order lattices (commonly referred to as standard LBM) with a limited
number of velocities, such as the D3Q19 lattice, is inherently restricted to
isothermal, weakly compressible flows due to the following two major
limitations:

    Numerical stability at high Mach numbers: For high Mach numbers,
    particularly in inviscid flows, the numerical stability of the method is
    constrained by the BGK collision model [69]. Its robustness is compromised
    by the emergence of non-hydrodynamic modes, which lead to instabilities [58,
    70]. Over the past few decades, significant efforts have been devoted to
    developing more stable collision operators [71]. Among these, the
    regularized and recursive regularized collision models, developed and
    utilized at M2P2, are discussed in more detail in Section 4.4.

    Mach-related errors and Galilean invariance: Due to the low-order velocity
    discretization, a Mach-related error—commonly referred to as the Galilean or
    symmetry-breaking error—arises in the momentum conservation equation. This
    error impedes the accurate modeling of temperature variations [58]. One
    approach to address this issue is the use of multi-speed lattices [58, 72],
    which increase the number of lattice velocities. However, this strategy
    reduces the computational efficiency advantage of the LB method and
    exacerbates instability issues by introducing additional non-hydrodynamic
    modes [73, 70].

The strategy adopted in this work is referred to as a segregated approach, where
the energy equation is solved separately. This can be achieved by introducing a
second set of distribution functions (the double distribution function approach)
or by coupling the LBM with a finite-volume formulation of the energy equation
(the hybrid approach). In this study, a segregated hybrid approach is employed,
and further details are provided in Section 4.5.

\subsection{Hybrid recursive regularised collision operator}
The BGK collision operator is the most simplest representation for the
collision, but suddenly it lacks from stability. One interesting thing in order
to improve stability is the recursive regularized step. 

\begin{equation}
	\Pi_{\alpha\beta}^{\bar{f}^{neq},(2)} = \sigma\sum_{i=0}^{q-1}c_{i\alpha}c_{i\beta}(\bar{f}_i - f_i^{eq} + \frac{\Delta t}{2}F_i) - (1-\sigma)\rho c_s^2 \bar{\tau}\left[ \frac{\partial u_\alpha}{\partial x_{\beta}} + \frac{\partial u_\beta}{\partial x_{\alpha}} -\frac{2\delta_{\alpha\beta}}{3}\frac{\partial u_\gamma}{\partial x_{\gamma}} \right]_{FD}
\end{equation}

In this methods we combine the Lattice Boltzmann method with the finite
difference. The viscous stress tensor is going to be computed from the LBM and
mixed with his equivalent part from a finite difference method. As pointed in
Jacob's work, make this method improves stability due a generation of an
artificial hyperviscoity.

\subsection{Energy Equation Strategy}

This section details the strategy employed to solve the total energy
conservation equation. As discussed in Section 4.3.3, the chosen approach
belongs to the segregated hybrid family. This decision is supported by recent
studies demonstrating the effectiveness of such methods for simulating
compressible flows [76–82]. In particular, Zhao et al. [67] highlighted the
value of stability and accuracy-preserving methods, noting that coupling an LB
scheme with an explicit discrete energy equation can introduce new instabilities
[83].

In the context of supercritical fluids, it is well-known that spurious pressure
waves generated by these instabilities can significantly impact the flow [7, 42,
43, 84]. The model adopted in this work, developed by Wissocq et al. [66],
leverages the Riemann invariants of the hyperbolic system formed by the Euler
equations. By design, this method ensures compatibility with LB fluxes,
resulting in a conservative total energy scheme that is linearly equivalent to
its non-conservative entropy-based counterparts. A detailed description of this
model is provided in Section 4.5.1.

However, in the presence of strong gradients, high-order methods used to
discretize the total energy scheme may induce Gibbs oscillations [85]. Following
the approach outlined in [66], this issue is addressed using slope limiters.
These limiters locally adjust the flux by blending high-order and first-order
fluxes, effectively damping spurious oscillations. To ensure low-order methods
are applied only where necessary, a carefully chosen limiter is essential. The
methodology for selecting and implementing these limiters is described in
Section 4.5.2.

\subsubsection{LBM/FV Coherence}

In this section, Wissocq’s total energy scheme [66] is outlined, with particular
emphasis on its construction procedure. To maintain clarity and avoid
unnecessary complexity, mathematical formulations involving lattice sets are not
explicitly presented. Readers seeking a deeper understanding are encouraged to
consult the original article for comprehensive details.

\subsubsection{Targeted continous equations}

In the inviscid case, the conservative system corresponds to the 3D Euler
equations, augmented by a conservative formulation of the total energy
conservation equation. This system can be expressed as:

\begin{equation}
	\partial_t U + \partial_\alpha F_\alpha^{U,c} = 0
\end{equation}

Where \( U = [\rho, \rho u_\alpha, \rho E]^T \) is the vector of conservative
variables, and \( F_\alpha^{U,c} = [\rho u_\alpha, \rho u_\alpha u_\beta +
\delta_{\alpha\beta}, \rho h_t u_\alpha] \). Here, \( F_\alpha \) represents the
flux vector in the direction \(\alpha\), and \( h_t = e_t + \frac{p}{\rho} \)
denotes the specific total enthalpy.

In most of the lattice Boltzmann method (LBM) literature, this system is
reformulated as a non-conservative one, where entropy serves as the energy
variable. 


\begin{equation}
	\partial_t V + A_{\alpha}^V \partial_{\alpha}F_\alpha^{V,c} = 0
\end{equation}

with $V = [\rho, \rho u_\alpha, s]^T$


\begin{equation}
	A_{\alpha}^V = 
	\begin{bmatrix}
		1 & 0 & 0 & 0 & 0 \\
		0 & 1 & 0 & 0 & 0 \\
		0 & 0 & 1 & 0 & 0 \\
		0 & 0 & 0 & 1 & 0 \\
		0 & 0 & 0 & 0 & u_\alpha
		\end{bmatrix}
\end{equation}

and $F_\alpha^{V,c} = [\rho u_\alpha, \rho u_\alpha u_\beta +
\delta_{\alpha\beta, s}]^T$

Since entropy is an intrinsic characteristic of the system [85], it follows a
simple advection equation. Moreover, assuming linearity, the entropy equation
becomes entirely decoupled from the LB scheme, ensuring that it does not impact
the scheme's stability [83].

\subsection{Discrete hybrid system based on the entropy equation}

In [66], Wissocq demonstrates that LB algorithms can systematically be
reformulated in a conservative form as:

\begin{equation}
	\partial_t \rho + \partial_\beta F^{\rho}_{+\Delta\beta/2} = 0
\end{equation}

\begin{equation}
	\partial_t (\rho u_\alpha) + \partial_\beta F^{\rho u_\alpha}_{+\Delta\beta/2} = 0
\end{equation}

Here, $F_{+\Delta \beta/2}^{\phi}(x,t)$ represents the intercell flux of the
dummy variable $\phi$ across the cell face located between the cells centered at
$x$ and $x+e_\beta\Delta x$, where $e_\beta$ is the unit vector in the $\beta$-direction. The
operators $\delta_t$ and $\delta_\beta$ denote the first-order temporal and spatial
derivatives (along direction $\beta$), respectively. Assuming the following explicit
discretization of the entropy advection equation: 

\begin{equation}
	\delta_t s + u_\alpha(x,t)\delta_\alpha^*s = 0
\end{equation}

Here, $\delta_\alpha^*$ is a spatial gradient operator associated with an arbitrarily
chosen numerical scheme. This operator can be reformulated using the spatial
gradient operator defined in Equation (4.35) as:

\begin{equation}
	\delta_\alpha^*s = \delta_\alpha\mathcal{F}^*_{+\Delta\alpha/2}(s)
\end{equation}

$\mathcal{F}^*_{+\Delta\alpha/2}$ being a linear function of s. The complete
discrete system then becomes

\begin{equation}
	\delta_t V + A_\alpha^V\delta_\alpha F_\alpha^{V,d} = 0
\end{equation}

Here, $F_\alpha^{V,d} = [F_{\Delta\alpha/2}^{\rho} , F_{\Delta\alpha/2}^{\rho
u_\beta}, \mathcal{F}^*_{+\Delta\alpha/2}(s)]$ represents the flux
contributions, but the system is not yet conservative as it describes the time
evolution of entropy rather than total energy as the energetic variable.
Additionally, $A_\alpha^V$ appears as a coefficient in front of the gradient
operator, which modifies the flux dynamics.

Finally, it is important to note that some degrees of freedom may remain in the
construction of LB fluxes, as there are fewer conserved variables in UU than
distributions in the LB solver.

\subsection {Conservative hybrid scheme}

In this section, a conservative hybrid scheme is derived from the generic
numerical scheme used for the entropy advection equation. This formulation
ensures that the linear stability of the energy equation does not interfere with
the mass and momentum equations.

The previously introduced discrete entropy system (Eq. 4.38) is multiplied by
the Jacobian matrix, which facilitates the transformation from non-conserved to
conserved variables, as follows:

\[
\mathbf{M} = \frac{\partial \mathbf{U}}{\partial \mathbf{V}} =
\begin{bmatrix}
1 & 0 & 0 & 0 & 0 \\
0 & 1 & 0 & 0 & 0 \\
0 & 0 & 1 & 0 & 0 \\
0 & 0 & 0 & 1 & 0 \\
h - \kappa & u_x & u_y & u_z & \rho T
\end{bmatrix}
\]

The kinetic energy of the system is denoted as $\kappa=u_\alpha^2/2$. The
non-conservative system described by Equation (4.38) can then be expressed as:

\begin{equation}
	\partial_t U + MA_\alpha^V\partial_\alpha F_\alpha^{V,d} = 0;
\end{equation}

This scheme is not yet in the conservative form because \( MA_\alpha^V \) is not
the identity matrix. What are the differences between our system and a
discretized form of equation (4.30) then? It is worth noting that only the last
line, corresponding to the conservation equation for the energetic variable, is
changed compared to the original non-conservative system. 

To understand how this equation changes, one can first multiply the continuous
non-conservative equation (4.32) by the passage matrix \( M \) and subtract the
result from equation (4.30). This leads to:

\begin{equation}
    \partial_\alpha F_\alpha^{U,c} = MA_\alpha^V \partial_\alpha F_\alpha^{V,c}
\end{equation}

Discretizing the gradient operator and combining the result with the last line
of the system described by equation (4.40) yields:

\begin{equation}
    \partial_t(\rho e_t) + \partial_\alpha^*(\rho h_t u_\alpha) 
    + (h-\kappa)\partial_\alpha[F_{\Delta\alpha/2}^\rho - \mathcal{F}_{\Delta\alpha/2}^*(\rho u_\alpha)]
    + u_\beta\delta_\alpha[F_{\Delta\alpha/2}^{\rho u_\beta} - \mathcal{F}_{\Delta\alpha/2}^*(\rho u_\alpha u_\beta + p\delta_{\alpha\beta})] = 0
\end{equation}

Incorporating the prefactors \((h - \kappa)\) and \( u_\beta \) into the
operator \(\delta_\alpha\) yields the desired conservative scheme. This approach
does not affect the consistency or linear properties of the system, as
demonstrated in the original article [66]. 

Furthermore, as shown by the original authors, this scheme restores the
consistency between the energy equation and the lattice Boltzmann (LB) scheme
used for mass and momentum equations. Finally, in the case of an extension to
viscous flows, this scheme naturally incorporates the viscous heating term, as
expected when solving the Navier-Stokes equations.


\subsection{Extension to viscous flows}

The discrete conservative form of the total energy equation for the Euler system
can be expressed as:

\begin{equation}
    \partial_t(\rho e_t) + \partial_\alpha F_{\Delta\alpha/2}^{\rho e_t} = 0
\end{equation}

On the other hand, the continuous total energy equation, considering viscous
heating and thermal conduction, reads:

\begin{equation}
    \partial_t(\rho e_t) + \partial_\alpha (\rho h_t u_\alpha) = \partial_\alpha (u_\beta \Pi_{\alpha\beta}^{NS}) + \partial_\alpha (\lambda \partial_\alpha T)
\end{equation}

Here, \( \Pi_{\alpha\beta}^{NS} \) is the Navier-Stokes viscous stress tensor.
As mentioned earlier, only the heat conduction term requires explicit
discretization to extend the scheme to the Navier-Stokes case. This extension is
achieved by replacing the Euler flux \( F_{\Delta\alpha/2}^{\rho e_t} \) with
its Navier-Stokes counterpart:

\begin{equation}
    F_{\Delta\alpha/2}^{\rho e_t, NS} = F_{\Delta\alpha/2}^{\rho e_t} - \lambda \partial_\alpha T(x + c_i^\alpha \Delta x, t)
\end{equation}


\subsection{important properties of the energy scheme}

\subsection{Extension of the scheme with force terms}
Until now, we have a solver that can solve the LBM coupled with a finite volume
solver for the energy. This method was developed by Wissocq et al.
\cite{wissocq2022restoring} but in this demonstration the force term was not
taked in account. 

As stated before, in the introdction, the objective of this thesis is to couple
this compressible solver with the Immersed Boundary Method and use that to
simulate supercritical fluids. 

The IBM, explained in a simple way is the representation as a source therm
(Source of momentum of energy) of the solid when has an interaction with the
fluid. Also the energy scheme proposed in \cite{wissocq2022restoring} serve to
recorver the conservativity of all the equations, otherwise the scheme is
loosing energy and mass.

The main change is that the collision invariant is not longer true, therefore,
another strategy should be taked in account in orther to recover the flux.

To develop that, we are going to start from the variable change in order to
recover a second orther algorithm.

\begin{equation}
	\bar{f}_i(x,t+\Delta t) = {f}_i(x,t+\Delta t) - \frac{\Delta t}{2}\left(\Omega_i(x,t+\Delta t) + F_i(x,t+\Delta t)\right) 
\end{equation}

The we take the first moment of this equation.

\begin{equation}
	\sum{c_i\bar{f}_i(x,t+\Delta t)} = \rho u_\alpha(x,t+\Delta t) - \frac{\Delta t}{2}F_\alpha(x,t+\Delta t)
\end{equation}

\begin{equation}
	\rho u_\alpha(x,t+\Delta t) = \sum{c_i\bar{f}_i(x,t+\Delta t)} + \frac{\Delta t}{2}F_\alpha(x,t+\Delta t)
\end{equation}

from this last equation and using the fact due to the streamming operation
$f_i^{col}(x-c_i\Delta t, t) = \bar{f}_i(x,t+\Delta t)$, we have:

\begin{equation}
	\rho u_\alpha(x,t+\Delta t) = \sum{c_i f_i^{col}(x-c_i\Delta t, t)} + \frac{\Delta t}{2}F_\alpha(x,t+\Delta t)
\end{equation}

then substracting $\rho u_\alpha(x,t)$ in each member of the last equation we have:

\begin{equation}
	\rho u_\alpha(x,t+\Delta t) -\textcolor{red}{\rho u_\alpha(x,t)}= \sum{c_i f_i^{col}(x-c_i\Delta t, t)} + \frac{\Delta t}{2}F_\alpha(x,t+\Delta t) -\textcolor{red}{\rho u_\alpha(x,t)}
\end{equation}

Also we known that:

\begin{equation}
	\rho u_\alpha(x,t) = \sum{c_i{f}_i^{col}(x,t)} - \frac{\Delta t}{2}F_\alpha(x,t)
\end{equation}

Then we have:

\begin{equation}
	\rho u_\alpha(x,t+\Delta t) -\textcolor{red}{\rho u_\alpha(x,t)}= \sum{c_i f_i^{col}(x-c_i\Delta t, t)} + \frac{\Delta t}{2}F_\alpha(x,t+\Delta t) -\textcolor{red}{\sum{c_i{f}_i^{col}(x,t)} + \frac{\Delta t}{2}F_\alpha(x,t)}
\end{equation}

\begin{equation}
	\rho u_\alpha(x,t+\Delta t) -\textcolor{red}{\rho u_\alpha(x,t)}= \sum{c_i f_i^{col}(x-c_i\Delta t, t)} -\textcolor{red}{\sum{c_i{f}_i^{col}(x,t)}}+ \frac{\Delta t}{2}F_\alpha(x,t+\Delta t) + \textcolor{red}{\frac{\Delta t}{2}F_\alpha(x,t)}
\end{equation}

Finally:

\begin{equation}
	\rho u_\alpha(x,t+\Delta t) =\textcolor{red}{\rho u_\alpha(x,t)} + \sum{c_i f_i^{col}(x-c_i\Delta t, t)} -\textcolor{red}{\sum{c_i{f}_i^{col}(x,t)}}+ \frac{\Delta t}{2}\left(F_\alpha(x,t+\Delta t) + \textcolor{red}{{F_\alpha(x,t)}}\right)
\end{equation}