\addchap{Conclusion}

This dissertation has addressed the development and application of advanced
numerical methods for the simulation of supercritical carbon dioxide (s\ce{CO2})
flows, with a focus on their integration into an innovative machine concept. The
motivation for this work stems from the growing interest in s\ce{CO2} power cycles as
a promising pathway to high-efficiency, compact, and sustainable energy
conversion systems. However, accurate modeling of s\ce{CO2} flows presents a number
of challenges: sharp variations of thermodynamic properties near the critical
point, the necessity for stable low-Mach formulations, and the need to handle
complex industrial geometries. To meet these challenges, a set of complementary
methodological contributions have been presented across five chapters of this
thesis, culminating in the demonstration of the proposed approach on an
industrially relevant test case.

\addsec{Synthesis of contributions}

The first chapter introduced the lattice Boltzmann method (LBM) as the
computational foundation of this work. LBM was chosen for its inherent
parallelism, flexibility in handling complex boundary conditions, and growing
maturity in simulating thermofluid phenomena. The study established the
mathematical framework and highlighted the potential of LBM as an alternative to
conventional Navier–Stokes solvers, thereby laying the groundwork for subsequent
developments.

In the second chapter, emphasis was placed on the incorporation of realistic
thermodynamics. The accurate description of thermodynamic properties is
essential when dealing with s\ce{CO2}, especially near the pseudo-critical line where
density and compressibility can vary dramatically with small temperature or
pressure perturbations. A robust and efficient equation-of-state implementation
was developed and coupled to the LBM framework, enabling simulations that
faithfully capture the unique behavior of s\ce{CO2}. This contribution was critical
in bridging the gap between abstract numerical methods and the physical fidelity
required for engineering applications.

The third chapter extended the methodology to complex geometries using immersed
boundary methods (IBM). Industrial machines rarely consist of simple channels;
their performance depends strongly on interactions between the working fluid and
intricate blade passages, cavities, and cooling channels. By implementing IBM
within the LBM framework, the thesis demonstrated the ability to simulate flows
around geometrically complex domains without resorting to excessive meshing
overhead. This methodological advance ensured that the simulation framework
could be deployed in practical design scenarios.

Chapter four focused on a novel low-Mach number formulation of the LBM, tailored
for supercritical \ce{CO2} applications. Standard LBM formulations often suffer from
stability and accuracy issues at low Mach regimes typical of turbomachinery and
heat exchanger applications. The newly developed formulation preserved the
advantages of LBM while ensuring robustness for s\ce{CO2} flows. This advance
represents a key step forward in making LBM a reliable tool for the study of
compressible yet subsonic regimes relevant to power cycle components.

Finally, the fifth chapter validated the framework through an industrial test
case. The methodology was applied to a realistic machine configuration,
demonstrating the feasibility of simulating s\ce{CO2} under operating conditions
representative of practical systems. This chapter not only showcased the
applicability of the proposed approach but also served as a proof of concept for
its future adoption in industrial design and optimization.

\addsec{Key Achievements}

Taken together, the work presented in this thesis delivers several contributions
to the field of computational thermofluids:  
\begin{itemize}
    \item Development of an LBM-based framework capable of simulating
    supercritical CO$_2$ with realistic thermodynamics.  
    \item Integration of immersed boundary methods to handle complex geometries
    relevant to turbomachinery.  
    \item Formulation of a stable and accurate low-Mach version of LBM
    applicable to supercritical fluids.  
    \item Demonstration of the methodology on an industrially relevant machine,
    bridging numerical method development with applied engineering challenges.  
\end{itemize}

These contributions represent a step toward reliable, efficient, and versatile
simulation tools that can accelerate the adoption of sCO$_2$ technologies in
energy conversion systems.  

\addsec{Limitations}

Despite these advances, several limitations must be acknowledged. The
computational cost of high-resolution LBM simulations remains significant,
particularly when three-dimensional domains and fine thermodynamic resolution
are required. While the IBM offers flexibility in handling geometry, it can
introduce spurious effects near boundaries, which may limit accuracy in strongly
coupled fluid--structure scenarios. Moreover, the low-Mach formulation, though
robust, may require further validation against a broader set of experimental
data, particularly in highly transient or off-design conditions. Finally, the
industrial case presented here represents an important first step, but further
scaling and coupling with system-level analyses remain necessary to fully
exploit the methodology in design practice.  

\addsec{Perspectives and Outlook}

The findings of this dissertation open several promising avenues for future
research. From a methodological perspective, the extension of the framework to
include turbulence modeling, conjugate heat transfer, and possibly multiphase
effects (e.g., condensation of CO$_2$ in transcritical regimes) would broaden
its applicability. Adaptive mesh refinement within the LBM framework could help
mitigate computational cost while preserving accuracy near critical regions.  

From an application perspective, further validation against experimental data
and comparison with industrial CFD tools will be essential for industrial
adoption. Coupling the solver with optimization algorithms could enable the
direct use of the framework in design cycles, providing insights not only into
performance but also into off-design behavior, safety margins, and long-term
reliability.  

More broadly, the methodology developed here contributes to the ongoing efforts
toward cleaner and more efficient energy systems. By advancing the numerical
modeling of sCO$_2$ flows, this work supports the engineering of next-generation
power cycles, offering potential benefits in compactness, efficiency, and
sustainability.  

\addsec{Closing Remarks}

In conclusion, this thesis has presented a coherent set of methodological
advances that enable the simulation of supercritical CO$_2$ flows in complex
machine configurations. Starting from the fundamentals of the lattice Boltzmann
method, through the incorporation of realistic thermodynamics and boundary
treatments, to the development of a dedicated low-Mach formulation and its
application to an industrial case, the research has built a bridge between
computational innovation and engineering application. While challenges remain,
the work lays the foundation for future exploration and industrial deployment of
sCO$_2$ simulations, with the ultimate goal of enabling more efficient and
sustainable energy conversion technologies. 
