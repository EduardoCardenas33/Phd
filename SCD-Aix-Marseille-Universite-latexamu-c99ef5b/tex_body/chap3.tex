\chapter{ Immersed Boundary Methods}

\section{Overview of the Immersed Boundary Method in CFD}
The Immersed Boundary (IB) Method is a computational approach in CFD used to
simulate fluid-structure interactions and complex boundary problems. It offers
flexibility in handling interfaces between fluids and solid structures, which
are represented within a fixed (usually Cartesian) grid, rather than requiring
body-fitted grids that conform to complex geometries. The IB method was
pioneered to study biological flows and has since expanded to a wide range of
fields including engineering, medicine, and environmental sciences. Below, we
trace the key developments that have shaped the evolution and application of the
IB method.

\subsection{Origins and Early Development: Peskin’s Work (1970s)}
The IB method was first introduced by Charles Peskin in the early 1970s to model
blood flow through heart valves. Peskin's approach was innovative in that it
allowed for the simulation of moving boundaries (such as the flexible heart
valves) within a fixed grid. Traditionally, simulating complex structures would
require deforming the mesh to match the structure, which is computationally
intensive and can lead to numerical instability.

Peskin's original IB method employed a two-phase approach:

\begin{enumerate}
    \item A Lagrangian representation for the boundary or interface, where
    discrete points represented the boundary, allowing for the flexibility of
    movement.
    \item An Eulerian grid for the fluid domain, which remained fixed in space.
\end{enumerate}

The Lagrangian points interact with the Eulerian grid through a smoothing
function, which distributes forces from the boundary to the fluid domain and
interpolates fluid velocities back to the boundary points.

\section{General Formulation}
Para ilustrar de manera grafica, la siguiente figura, muestra un esquema de como
se representan la parte fluida y solida en el metodo de fronteras sumergidas de
manera discreta.

En este esquema, se designa la parte fluida $\omega_f$ como nodos eulerianos
representados en color negro, mientras que la parte solida $\omega_s$ es
designada como Lagrangiana y en color rojo.  

Otra convecion que utilizada en la presente documento es la de utilizar letras
mayusculas para la parte Lagrangiana y minusculas para la parte euleriana.

En general, el metodo IB consiste en la representacion del solido por medio de
terminos fuente, ya sea de cantidad de movimiento o de energia. Es decir, en
lugar de representar completamente el solido, solo se veran sus efectos en los
puntos Lagrangianos. Estos efectos, luego seran repartidos hacia el campo
euleriano mediante una operacion conocida como Spreading.

\subsection{Source term in momentum equation}

This method originally was developed using the Navier Stokes equations. As a
first approach and to have a good comprehension of the method, a brief
demonstration using the classical Navier Stokes equations is showed. Then, a
similar development is developed but using the Lattice Boltzmann Method. After
making that, we can profite an reduce the algorithme using some properties and
characteristiques of the LBM.

\subsubsection{Similar to Navier Stokes developments}

La ecuacion general de la conservacion de la cantidad de movimiento en un fluido puede ser
descrita de la siguiente manera;

\begin{equation}
	\frac{\partial \rho u_\alpha}{\partial t} + \frac{ \partial(\rho u_\alpha u_\beta + p\delta_{\alpha\beta}-\Pi_{\alpha\beta})}{\partial x_\beta} = F
\end{equation}

\begin{equation}
	\frac{\partial \rho u_\alpha}{\partial t}  = F + RHS
\end{equation}

De esta ecuacion, podemos decir que el termino de fuerza representa la influencia del solido,
en el dominio fluido. Sin embargo por el momento este termino es desconocido.

Para obtener la funcion matematica que permite calcular el termino de fuerza, el
metodo se divide en dos etapas. Una primera etapa donde se realiza una
simualcion sin tomar en cuenta el termino fuerza, esta etapa es conocida como la
etapa de prediccion.

\begin{equation}
	\frac{\partial \rho u_\alpha}{\partial t}  = RHS
\end{equation}

Discretizando la ecuacion tenemos:
\begin{equation}
	\frac{ (\rho u)^* - (\rho u)^n}{\Delta t}  = RHS
\end{equation}

Si discretizamos la ecuacion original tenemos:
\begin{equation}
	\frac{ (\rho u)^{n+1} - (\rho u)^n}{\Delta t}  = RHS + F
\end{equation}

Si quisieramos que en un solo paso de tiempo, la velocidad alcance el valor de
la velocidad target, la expresion se transforma en: 

\begin{equation}
	\frac{ (\rho u)^{t} - (\rho u)^n}{\Delta t}  = RHS + F
\end{equation}

Finalmente tenemos que:

\begin{equation}
	F = \frac{(\rho u)^{t} - (\rho u)^*}{\Delta t}
\end{equation}


En resumen, para aplicar el metodo IB y poder determinar la influencia del
solido en el fluido, debemos aplicar dos veces la ecuacion de conservacion de
momento de mecanica de fluidos. Una primera vez para poder hacer una prediccion
y una segunda para poder hacer una correcion aplicando el termino de fuerza.





\section{Choice of interpolation kernel}
En la literatura existen diversos kernel de interpolacion, tales como los
propuestos por Peskin, Roma, etc. Estos Kernel deben de cumplir con ciertas
condiciones para poder permitir una buena interpoalcion y spreading para la
comunicacion entre el dominio fluido y el dominio solido. La principal
condicion es que el kernel sea de clase uno, $C_1$ al menos.

\begin{equation}
	\delta_h(x) = \frac{1}{h^3}\phi\left(\frac{x}{h}\right)\phi\left(\frac{y}{h}\right)\phi\left(\frac{z}{h}\right)
\end{equation}

En el presente trabajo de tesis se utiliso, el kernel dado por Peskin, el cual
tiene la forma siguiente:

\begin{equation}
	\phi(r) = \frac{1}{4}(1+cos(\frac{\pi r}{2}))
\end{equation}


\section{Weight calibration}
Como descrito anteriormente, la region solida definida de acuerdo al metodo de
fronteras sumergidas, tiene una dimension menos que la dimension del dominio
fluido. Esto provoca que no exista una reprocidad directa entre el campo
euleriano y lagrangiano, por ello, una calibracion es necesaria de realizar de
tal manera que existq una reciprocidad y conservacion entre ambos espacios.

\begin{equation}
	W_k = \frac{1}{\sum_l\sum_j\delta(x_j-X_K)\delta(x_j-X_l)}
\end{equation}

\section{Linking LBM with LBM}
In order to link this two methods, we have to use two steps, the prediction and
the corrector step.

\subsection{Prediction}
In the prediction step, all the calculations are made without the forces. Using
equation (\ref{explicit equivalent equation}) we have:

\begin{equation}
    \Bar{f}_i^* (t+\Delta t, x_i + c_i \Delta t) = \Bar{f}_i (t, x) + \Delta t \, \Omega_i (t, x_i)
\end{equation}

It seems that in this equation we eliminate the forces but in reality there
exist some force terms hidden in $\Bar{f_i}$.\footnote{Another important  thing
is to note that this equation is second order because it comes from the Cranck
Nicholson step, we can use an upwind scheme also but it's only first order} 

Using the definition of $\Bar{f_i}$, replacing in equation 8 and eliminating
also the hidden force terms: 

\begin{equation}
    f_i^* (t+\Delta t, x_i + c_i \Delta t) 
    - \frac{\Delta t}{2} \, \Omega_i (t+\Delta t, x_i + c_i \Delta t) 
    = f_i (t, x) 
    + \frac{\Delta t}{2} \, \Omega_i (t, x_i).
\end{equation}

\begin{equation}
    f_i^* (t+\Delta t, x_i + c_i \Delta t) 
    = f_i (t, x) 
    + \frac{\Delta t}{2} \left( \Omega_i (t, x_i) 
    + \frac{\Delta t}{2} \, \Omega_i (t+\Delta t, x_i + c_i \Delta t) \right).
\end{equation}

\begin{equation}
    f_i^* (t+\Delta t, x_i + c_i \Delta t) = f_i (t, x) + \frac{\Delta t}{2} \, (\text{RHS})
    \label{Prediction equation}
\end{equation}

\subsection{Correction}
In this part we are going to use the forcing term exactly as in equation
(\ref{explicit equivalent equation}).  

\begin{equation}
    \Bar{f_i^t}_{(t+\Delta t, x_i +c_i \Delta t)} = \Bar{f_i}_{(t,x)} + \Delta t(\Omega_i +F_i)_{(t, x_i)}
\end{equation}

Using the definition of $\Bar{f_i}$ we have:

\begin{equation}
    \Bar{f_i} (t+\Delta t, x_i + c_i \Delta t) = f_i (t, x) + \frac{\Delta t}{2} \left( \Omega_i + F_i \right)_{(t, x_i)}
\end{equation}  

\begin{equation}
    f_i^t (t+\Delta t, x_i + c_i \Delta t) 
    - \frac{\Delta t}{2} \left( \Omega_i + F_i \right)_{(t+\Delta t, x_i + c_i \Delta t)} 
    = f_i (t, x) 
    + \frac{\Delta t}{2} \left( \Omega_i + F_i \right)_{(t, x_i)}.
\end{equation}  

\begin{equation}
    f_i^t (t+\Delta t, x_i + c_i \Delta t) = f_i (t, x) 
    + \frac{\Delta t}{2} \left( \Omega_i + F_i \right)_{(t, x_i)} 
    + \frac{\Delta t}{2} \left( \Omega_i + F_i \right)_{(t+\Delta t, x_i + c_i \Delta t)}.
\end{equation}  

\begin{equation}
    f_i^t (t+\Delta t, x_i + c_i \Delta t) = f_i (t, x) 
    + \frac{\Delta t}{2} \left( \Omega_i (t, x_i) + \Omega_i (t+\Delta t, x_i + c_i \Delta t) \right) 
    + \frac{\Delta t}{2} \left( F_i (t, x_i) + F_i (t+\Delta t, x_i + c_i \Delta t) \right).
\end{equation}

\begin{equation}
    f_i^t (t+\Delta t, x_i + c_i \Delta t) = f_i (t, x) 
    + \frac{\Delta t}{2} (\text{RHS}) 
    + \frac{\Delta t}{2} \left( F_i (t, x_i) + F_i (t + \Delta t, x_i + c_i \Delta t) \right)
    \label{Correction equation}
\end{equation}

Subtracting equation (\ref{Prediction equation}) from (\ref{Correction
equation}):

\begin{equation}
    f_i^t (t+\Delta t, x_i + c_i \Delta t) - f_i^* (t+\Delta t, x_i + c_i \Delta t) 
    = \frac{\Delta t}{2} \left( F_i (t, x_i) + F_i (t + \Delta t, x_i + c_i \Delta t) \right)
    \label{IBM with LBM}
\end{equation}

Now applying the reconstruction to the equation (\ref{IBM with LBM}) in order to
have the macroscopic properties: 

\begin{equation}
    \rho u^t - \rho u^* = \frac{\Delta t}{2} (F_{(t)} + F_{(t+\Delta t)})
    \label{Real IBM LBM}
\end{equation}

We can see that if the force is constant the equations from Navier Stokes and
LBM are equal.

In a similar way, if we use a first order scheme for equation (8), we are going
to arrive to the Navier Stokes equation.

\section{An one step algorithme}
As seen before, all the demonstration was deducted solving to times the
Boltzmann equation. One time in order to determine the required force and a
second time using this force to find the good velocity, pressure, temperature,
etc field.

This is the classical way that was herited from the Navier Stokes developements,
but one advantage of using the Immersed Boundary method in the Lattice Boltzmann
framework is that in order to determine the force, we don't need to make a
prediction.

Using the defition of the variable change made in order to have a second order
algorithme, we can have an expression to apprximate the required force that
represents the effect of the solid on the fluid.

From the variable change, we have:

\begin{equation}
    \bar{f_i} = f_i - \frac{\Delta t}{2}({\Omega + F_i})
\end{equation}

Taking the first moment od the last equation we have:
\begin{equation}
    \sum{c_i\bar{f}} = \rho u_\alpha - \frac{\Delta t}{2}({\sum F_i})
\end{equation}

This equation is valid for all the times steps and positions. Suddenly if we
would like to compute the force term we need to known the value of the first
moment of the bar distribution function, but this value is unknown because we
need the force term to solve the Boltzmann method. This is why, in order to
approximate the force term, the previous value of the first moment is used. 

Finally, the expression to approximate the force term is:

\begin{equation}
    F_\alpha = 2(\rho u_\alpha - \sum {c_i\bar{f_i}})
\end{equation}

In this manuscript, the one step algorithme is used due the gain in time because
the LBM is going to be solved only one time.

\subsection{Energy source term}

The total energy equation is:

\begin{equation}
    \frac{\partial\rho E}{\partial t} + \frac{\partial\left[(\rho E +p)u_\beta + q_\beta - u_\alpha\Pi_{\alpha\beta}\right]}{\partial x_\beta} = \dot{Q}_{IBM}
\end{equation}

In a similar way that was made for the momentum equation, we can estimate the
IBM source term in order to represent the impact of a solid on the fluid. To
make that, a prediction of the temperature should be made. 

In this case the prediction of the temperture on the solid is going to be only
related with the temperature of the fluid around the solid using for that the
interpolation operator.

\begin{equation}
    \rho E^t - \rho E^{n} = \dot{Q}_{IBM}
\end{equation}

Where $E = e + \frac{1}{2}\rho u^2$, then:

\begin{equation}
    \rho {e^t + \frac{1}{2}\rho u^t}^2 - \rho {e^n - \frac{1}{2}\rho u^n}^{2} = \dot{Q}_{IBM}
\end{equation}

\begin{equation}
    \rho {e^t} - \rho {e^n} + {\frac{1}{2}\rho u^t}^2  - {\frac{1}{2}\rho u^n}^{2} = \dot{Q}_{IBM}
\end{equation}

\begin{equation}
    \rho {e^t} - \rho {e^n} + \left({\rho u^t}  - {\rho u^n}\right) \left(\frac{u^t + u^n}{2}\right) = \dot{Q}_{IBM}
\end{equation}

\begin{equation}
    \rho {e^t} - \rho {e^n} + \left(F_{IBM}\right) \left(\frac{u^t + u^n}{2}\right) = \dot{Q}_{IBM}
\end{equation}

What is interesting to see here is the presence of the force IBM. This term
represent the work of this force.

Another important remark in this model is that this source term for the energy
doesn't need a prediction step, we are going to use directly the interpolated
internal energy and the interpoleted velocity. This makes that the algorithme
should be solved only once. 

\section{Numerical implementation}
\section{Coupling with Lattice Boltzmann equations}
\subsection{Algorithme}
\begin{enumerate}
	\item Displacement of Lagrangian points according to the chosen physical
    model for boundary deformation; update of $U_d(X,t+\Delta t)$ and the
    interpolation and spreading operators based on the new geometry.

    \item Calculation of fi(x,t)
    
    \item Calculation of the IBM force 

    \item Solution of the discretized Boltzmann equation (2.9), yielding \(
    f_i(x, t + \Delta t) \) at the next time step.

    \item Calculation of the density \( \rho(x, t + \Delta t) \).

    \item Interpolation of the terms in equation (2.26) at time \( t + \Delta t
    \), and evaluation of the force \( F(X, t + \Delta t) \) on the Lagrangian
    points based on the desired velocity \( U_d(X, t + \Delta t) \).

    \item Spreading of the force \( F(X, t + \Delta t) \) onto the neighboring
    Eulerian points, yielding \( f(x, t + \Delta t) \) via equation (1.3).

    \item Update of the fluid state variables \( \rho(x, t + \Delta t) \) and \(
    u(x, t + \Delta t) \) at time \( t + \Delta t \) using equation (2.14),
    which includes the forcing term; return to step 1 for the next time step.
\end{enumerate}