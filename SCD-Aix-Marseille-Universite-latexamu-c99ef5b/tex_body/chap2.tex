\chapter{ Thermodynamics}
\chaptertoc{}

\section{Ideal Gas EoS}

A mathematical relationship that connects temperature, pressure, and volume is known as an Equation of State (EOS). These equations can be expressed in either a pressure-explicit form (such as CEOS) or a volume-explicit form. The ideal gas (IG) model is the most basic theoretical representation of a gas, grounded in the principles of kinetic theory. This model is based on the key assumption that the gas consists of point-like particles, which interact only through elastic collisions, with no long-range intermolecular forces. The main advantage of this simplified model is its straightforward and easy-to-use EOS, which is described below:

\begin{equation}
	P = \frac{RT}{V}
\end{equation}

Where $P$ represents the gas pressure, $V$ is the molar volume, $T$ is the temperature, and $R$ is the ideal gas constant. This equation can also be expressed in a specific form as follows:

\begin{equation}
	P = \frac{rT}{v} = \rho rT
\end{equation}

Where $\rho = \frac{1}{v}=\frac{M}{V}$ represents the density, $M$ is the fluid's molecular weight, , $v$ is the specific volume of the fluid, and $r$ is known as the specific gas constant:

\begin{equation}
	r = \frac{R}{M}
\end{equation}

	\subsection{Heat capacities, enthalpy and entropy}

	Using the Ideal Gas Equation of State (IG EOS), the following standard relationships apply:
	
	\begin{enumerate}
		\item Mayer's relation:
		\begin{equation}
			C_P - C_V = R
		\end{equation}

		\item Variation in internal energy:
		\begin{equation}
			\mathrm{d}E = C_V\mathrm{d}T
		\end{equation}

		\item Variation in enthalpy:
		\begin{equation}
			\mathrm{d}H = C_P\mathrm{d}T
		\end{equation}

		\item Variation in entropy:
		\begin{equation}
			\mathrm{d}S = \frac{C_P}{T}\mathrm{d}T - R\frac{\mathrm{d}P}{P}
		\end{equation}

	\end{enumerate}

	\subsection{NASA Polynomials}
	
	In its simplest form, the Ideal Gas Equation of State (IG EOS) assumes constant heat capacities, leading to linear relationships between enthalpy and internal energy 
	with respect to temperature. However, this assumption is not valid across a wide range of temperatures. For example, in the field of combustion, the NASA polynomial 
	model is commonly used to describe heat capacities more accurately. The NASA-7 model, for instance, expresses heat capacities in the following form [10]:

	\begin{equation}
		\frac{C_p}{R} = a_1 + a_2 + a_3 T^2 + a_4 T^3 + a_5 T^4
	\end{equation}

	\begin{equation}
		\frac{H}{RT} = a_1 + \frac{a_2}{2}T + \frac{a_3}{3}T^2 + \frac{a_4}{4}T^3 + \frac{a_5}{5}T^4 + \frac{a_6}{T}
	\end{equation}

	\begin{equation}
		\frac{S}{R} = a_1 \ln T + a_2 T + \frac{a_3}{2}T^2 + \frac{a_4}{3}T^3 + \frac{a_5}{4}T^4 + a_7
	\end{equation}
	The ideal gas equation of state (EOS) is a straightforward yet accurate model for describing gas behavior at high temperatures and relatively low pressures. 
	However, it falls short in representing condensed fluids. Despite this limitation, the ideal gas model has been extensively studied and documented over the years. 
	Consequently, many thermodynamic problems transition from the Real Gas (RG) state to the Ideal Gas (IG) state using residual or departure functions (see Section 2.4.3.2)
	to leverage the wealth of existing knowledge associated with the IG model.


	\subsection{Compressibility Factor}

	Real fluids at low density and high temperature can be accurately modeled by the perfect gas 
	equation of state (EOS). However, at lower temperatures or higher densities, a real fluid 
	deviates significantly from ideal gas behavior, especially during phase changes, such as when 
	it condenses from a gas to a liquid or deposits from a gas to a solid. This deviation is 
	quantified by the compressibility factor, $Z$, which indicates how much the fluid's behavior
	differs from that of an ideal gas:

	\begin{equation}
		Z = \frac{PV}{RT} =\frac{Pv}{rT} = \frac{P}{\rho RT} 
	\end{equation}

\section{Supercritical Fluids}

Supercritical phenomena are crucial in various industrial applications,
particularly in power cycles and energy recovery, which are the focus of this
work. This section is dedicated to understanding the key differences between
supercritical and subcritical fluids, with an emphasis on their role in power
generation.

Supercritical fluids, especially supercritical carbon dioxide ($CO_2$) and water,
are increasingly important in the development of advanced power cycles due to
their exceptional thermodynamic properties. These fluids are used to enhance
efficiency, reduce the size of equipment, and minimize environmental impact in
power generation systems. Here are some key applications:

    Supercritical $CO_2$ Power Cycles: Brayton Cycle: Supercritical $CO_2$ is a
        promising working fluid in closed-loop Brayton cycles, where it operates
        at temperatures and pressures above its critical point. These cycles
        have the potential to surpass the efficiency of traditional steam
        Rankine cycles, primarily because supercritical $CO_2$ has a higher
        density, which allows for more compact and efficient turbomachinery.
        This increased efficiency translates into lower fuel consumption and
        reduced greenhouse gas emissions, making supercritical $CO_2$ Brayton
        cycles a leading candidate for next-generation power plants. Waste Heat
        Recovery: One of the significant advantages of supercritical $CO_2$ cycles
        is their ability to recover waste heat from industrial processes, gas
        turbines, and even nuclear reactors. By utilizing waste heat, these
        systems can generate additional electricity without burning extra fuel,
        improving the overall efficiency of energy systems and reducing the
        carbon footprint of power generation.

    Supercritical Water Power Cycles: Supercritical Water-Cooled Reactors
        (SCWRs): In nuclear power generation, supercritical water is used as a
        coolant in advanced reactor designs. SCWRs operate at supercritical
        pressures and temperatures, allowing for higher thermal efficiency
        compared to conventional reactors. This not only improves the power
        output but also contributes to the safety and economic viability of
        nuclear energy.

    Supercritical Steam Cycles: Traditional fossil fuel power plants are
        evolving with the adoption of supercritical and ultra-supercritical
        steam cycles. By operating at temperatures and pressures above the
        critical point of water, these power plants achieve higher thermal
        efficiencies, reducing fuel consumption and emissions. The move towards
        supercritical steam cycles represents a significant advancement in
        making coal-fired power generation cleaner and more efficient.
		
    Hybrid Power Cycles: Integration with Renewable Energy: Supercritical $CO_2$
        and water cycles are being integrated with renewable energy sources,
        such as concentrated solar power (CSP) and geothermal energy. In CSP
        plants, supercritical $CO_2$ is used to transfer heat from the solar field
        to the power block, allowing for more efficient energy conversion.
        Similarly, in geothermal systems, supercritical fluids can enhance
        energy extraction from deep, high-temperature reservoirs, providing a
        reliable and sustainable source of electricity. Combined Cycle Plants:
        In combined cycle power plants, supercritical $CO_2$ can be used in
        conjunction with gas turbines to further increase overall efficiency.
        The exhaust heat from the gas turbine is used to generate additional
        power in a supercritical $CO_2$ cycle, making these plants among the most
        efficient forms of fossil fuel-based power generation.

    Environmental Benefits and Future Potential: Carbon Capture and Utilization:
        Supercritical $CO_2$ power cycles not only improve efficiency but also
        facilitate carbon capture and utilization (CCU). By integrating CCU with
        supercritical $CO_2$ cycles, power plants can reduce their carbon emissions
        while potentially producing valuable products like synthetic fuels. This
        dual benefit supports the transition to more sustainable energy systems.
        Future Research and Development: Ongoing research into supercritical
        fluid power cycles is focused on optimizing system design, improving
        materials that can withstand the harsh conditions of supercritical
        operation, and integrating these cycles with emerging technologies. The
        goal is to create highly efficient, low-emission power plants that can
        meet the growing global demand for electricity while minimizing
        environmental impact.

In conclusion, supercritical fluids are at the forefront of power generation
technology, offering significant advantages in efficiency, compactness, and
environmental sustainability. As research and development continue, these
systems are expected to play a central role in the future of energy, making them
a critical area of study for advancing power cycles and reducing the carbon
footprint of electricity generation.

In order to star with a better comprehension of the supercritical state
comprehension, some reduced parameters are introduced:

\begin{equation}
	Pr = \frac{P}{P_c}
\end{equation}

\begin{equation}
	T_r = \frac{T}{T_c}
\end{equation}

\begin{equation}
	V_r = \frac{V}{V_c}
\end{equation}

\begin{equation}
	\rho_r = \frac{\rho}{\rho_c}
\end{equation}

where subscript $c$ denotes the parameter's value at the critical point.

	\subsection{Physical Structure of Supercritical Fluids}

	The first topic to address is the nature of supercritical fluids. It is
	well-known that, in the supercritical state, there is no distinctly separate
	phase (see Figure 2.1) [14].

	Using molecular dynamics simulations of liquid and transcritical states at a
	reduced temperature of 0.5 and reduced pressures of 0.7, 1.4, and 3.0,
	Banuti [15] demonstrated that there is no need to separate quadrants IL and
	IV in Figure 2.1. This is because densely packed fluids in IL exhibit the
	same physical behavior as their supercritical counterparts in IV. For gases,
	however, the equivalency between the supercritical fluid and the ideal gas
	model in the ($P_r,T_r$) plane is reduced to a line. It is important to
	note that, up to $P_r=3$ and for $T_r>2$, describing a supercritical
	fluid as an ideal gas results in a compressibility factor error of less than
	5$\%$, as shown in Figure 2.2.

	Consequently, a pseudo-transition should exist, dividing the supercritical
	region, as illustrated in Figure 2.1. Recent independent studies by Gorelli
	et al. [17] and Simeoni et al. [18], using X-ray scattering, have shown that
	sound dispersion, typically observed only in liquids, indeed occurs in a
	supercritical transition. This suggests that there are two distinct
	pseudo-states within supercritical matter, implying the existence of a
	mechanism analogous to a subcritical phase transition.

	\subsection{Pseudo-Boiling} % entre [] pour le texte dans la TOC

		Ajout d'une nouvelle entrée d'index de la centrifugeuse\index{centrifugeuse}. Les entrées \gls{+a} \gls{2a} \gls{ca} \gls{Aa} \gls{aa} \gls{alpha} {\NoAutoSpaceBeforeFDP}sont dans la nomenclature. On peux utiliser les commandes personnelles pour appeler rapidement des formules lors de la rédaction \acc et passer des arguments aux commandes pour en modifier l'éxécution \emiss[\nu]{\Omega}.
		
		\subsubsection{Ce titre de partie ne s'affiche pas dans la TOC (tocdepth=2) mais dans la TOC locale (etocsettocdepth=3)}

			Voir (Tableaux~\ref{table:alpha}~et~\ref{table:butcher}).

			\paragraph{Ce titre de partie n'est pas numéroté (secnumdepth=3)}~~\\ % ~~\\ fait le saut de ligne après le titre de 'paragraph' sinon le texte suivant est accolé au titre

				% ce saut de ligne indente le texte de 'paragraph' sinon le paragraphe débute à la marge
				Ajout d'une citation entre parenthèses~\parencite{godard_borreliose_2012} avec la commande \textit{\textbackslash parencite}. Ajout d'une citation simple de \cite{zohdy_mapping_2012} avec la commande \textit{\textbackslash cite}. Ajout d'une citation avec année et page entre parenthèses de \textcite[9]{godard_borreliose_2012} avec la commande \textit{\textbackslash textcite}. La citation suivante, sur la même page, de \textcite[12]{godard_borreliose_2012} utilise ibidem avec le style de citation \textit{authoryear-ibid} et l'utilisation des options biblatex \textit{pagetracker} et \textit{ibidtracker}.

			\paragraph{Plusieurs figures côte à côte}~~\\

				\lipsum[66]


				\begin{figure}
					\centering
					\subfloat[Figure A]{\includegraphics[width=.5\textwidth, max height=2in]{example-image-a}}
					\subfloat[Figure B]{\includegraphics[width=.5\textwidth, max height=2in]{example-image-b}}
					\caption{Deux figures}
					\label{fig:deux_figures}
				\end{figure}

			\paragraph{Paramétrer siunitx avec sisetup}~~\\
			
        		Célérité de la lumière dans le vide: $$c=\SI{2.99792458e8}{\meter\per\second}$$
\section{Cubic EoS}
