\chapter{ Thermodynamics}
\chaptertoc{}

\section{Ideal Gas EoS}

A mathematical relationship that connects temperature, pressure, and volume is known as an Equation of State (EOS). These equations can be expressed in either a pressure-explicit form (such as CEOS) or a volume-explicit form. The ideal gas (IG) model is the most basic theoretical representation of a gas, grounded in the principles of kinetic theory. This model is based on the key assumption that the gas consists of point-like particles, which interact only through elastic collisions, with no long-range intermolecular forces. The main advantage of this simplified model is its straightforward and easy-to-use EOS, which is described below:

\begin{equation}
	P = \frac{RT}{V}
\end{equation}

Where $P$ represents the gas pressure, $V$ is the molar volume, $T$ is the temperature, and $R$ is the ideal gas constant. This equation can also be expressed in a specific form as follows:

\begin{equation}
	P = \frac{rT}{v} = \rho rT
\end{equation}

Where $\rho = \frac{1}{v}=\frac{M}{V}$ represents the density, $M$ is the fluid's molecular weight, , $v$ is the specific volume of the fluid, and $r$ is known as the specific gas constant:

\begin{equation}
	r = \frac{R}{M}
\end{equation}

	\subsection{Heat capacities, enthalpy and entropy}

	Using the Ideal Gas Equation of State (IG EOS), the following standard relationships apply:
	
	\begin{enumerate}
		\item Mayer's relation:
		\begin{equation}
			C_P - C_V = R
		\end{equation}

		\item Variation in internal energy:
		\begin{equation}
			\mathrm{d}E = C_V\mathrm{d}T
		\end{equation}

		\item Variation in enthalpy:
		\begin{equation}
			\mathrm{d}H = C_P\mathrm{d}T
		\end{equation}

		\item Variation in entropy:
		\begin{equation}
			\mathrm{d}S = \frac{C_P}{T}\mathrm{d}T - R\frac{\mathrm{d}P}{P}
		\end{equation}

	\end{enumerate}

	\subsection{NASA Polynomials}
	
	In its simplest form, the Ideal Gas Equation of State (IG EOS) assumes constant heat capacities, leading to linear relationships between enthalpy and internal energy 
	with respect to temperature. However, this assumption is not valid across a wide range of temperatures. For example, in the field of combustion, the NASA polynomial 
	model is commonly used to describe heat capacities more accurately. The NASA-7 model, for instance, expresses heat capacities in the following form [10]:

	\begin{equation}
		\frac{C_p}{R} = a_1 + a_2 + a_3 T^2 + a_4 T^3 + a_5 T^4
	\end{equation}

	\begin{equation}
		\frac{H}{RT} = a_1 + \frac{a_2}{2}T + \frac{a_3}{3}T^2 + \frac{a_4}{4}T^3 + \frac{a_5}{5}T^4 + \frac{a_6}{T}
	\end{equation}

	\begin{equation}
		\frac{S}{R} = a_1 \ln T + a_2 T + \frac{a_3}{2}T^2 + \frac{a_4}{3}T^3 + \frac{a_5}{4}T^4 + a_7
	\end{equation}
	The ideal gas equation of state (EOS) is a straightforward yet accurate model for describing gas behavior at high temperatures and relatively low pressures. 
	However, it falls short in representing condensed fluids. Despite this limitation, the ideal gas model has been extensively studied and documented over the years. 
	Consequently, many thermodynamic problems transition from the Real Gas (RG) state to the Ideal Gas (IG) state using residual or departure functions (see Section 2.4.3.2)
	to leverage the wealth of existing knowledge associated with the IG model.


	\subsection{Compressibility Factor}

	Real fluids at low density and high temperature can be accurately modeled by the perfect gas 
	equation of state (EOS). However, at lower temperatures or higher densities, a real fluid 
	deviates significantly from ideal gas behavior, especially during phase changes, such as when 
	it condenses from a gas to a liquid or deposits from a gas to a solid. This deviation is 
	quantified by the compressibility factor, ZZ, which indicates how much the fluid's behavior
	differs from that of an ideal gas:

	\begin{equation}
		Z = \frac{PV}{RT} =\frac{Pv}{rT} = \frac{P}{\rho RT} 
	\end{equation}

\section{Supercritical Fluids}

	Le faisceau passe ensuite dans un module comprenant un cristal non linéaire permettant de doubler le féquence (excitation de \SIrange{345}{500}{\nano\meter}). Toutes les mesures ont été faites entre \SIlist{400;1200}{\nano\meter} avec un pas de \SI{5}{\nano\meter}.
	
	\lipsum[3]\index{Nulla malesuada}

	\subsection{Physical Structure of Supercritical Fluids}

		\lipsum[4]\index{Quisque ullamcorper}

	\subsection{Pseudo-Boiling} % entre [] pour le texte dans la TOC

		Ajout d'une nouvelle entrée d'index de la centrifugeuse\index{centrifugeuse}. Les entrées \gls{+a} \gls{2a} \gls{ca} \gls{Aa} \gls{aa} \gls{alpha} {\NoAutoSpaceBeforeFDP}sont dans la nomenclature. On peux utiliser les commandes personnelles pour appeler rapidement des formules lors de la rédaction \acc et passer des arguments aux commandes pour en modifier l'éxécution \emiss[\nu]{\Omega}.
		
		\subsubsection{Ce titre de partie ne s'affiche pas dans la TOC (tocdepth=2) mais dans la TOC locale (etocsettocdepth=3)}

			Voir (Tableaux~\ref{table:alpha}~et~\ref{table:butcher}).

			\paragraph{Ce titre de partie n'est pas numéroté (secnumdepth=3)}~~\\ % ~~\\ fait le saut de ligne après le titre de 'paragraph' sinon le texte suivant est accolé au titre

				% ce saut de ligne indente le texte de 'paragraph' sinon le paragraphe débute à la marge
				Ajout d'une citation entre parenthèses~\parencite{godard_borreliose_2012} avec la commande \textit{\textbackslash parencite}. Ajout d'une citation simple de \cite{zohdy_mapping_2012} avec la commande \textit{\textbackslash cite}. Ajout d'une citation avec année et page entre parenthèses de \textcite[9]{godard_borreliose_2012} avec la commande \textit{\textbackslash textcite}. La citation suivante, sur la même page, de \textcite[12]{godard_borreliose_2012} utilise ibidem avec le style de citation \textit{authoryear-ibid} et l'utilisation des options biblatex \textit{pagetracker} et \textit{ibidtracker}.

			\paragraph{Plusieurs figures côte à côte}~~\\

				\lipsum[66]


				\begin{figure}
					\centering
					\subfloat[Figure A]{\includegraphics[width=.5\textwidth, max height=2in]{example-image-a}}
					\subfloat[Figure B]{\includegraphics[width=.5\textwidth, max height=2in]{example-image-b}}
					\caption{Deux figures}
					\label{fig:deux_figures}
				\end{figure}

			\paragraph{Paramétrer siunitx avec sisetup}~~\\
			
        		Célérité de la lumière dans le vide: $$c=\SI{2.99792458e8}{\meter\per\second}$$
\section{Cubic EoS}
