\chapter{ Thermodynamics}
\chaptertoc{}

\section{Ideal Gas EoS}

A mathematical relationship that connects temperature, pressure, and volume is
known as an Equation of State (EOS). These equations can be expressed in either
a pressure-explicit form or a volume-explicit form. The ideal gas
(IG) model is the most basic theoretical representation of a gas, grounded in
the principles of kinetic theory. This model is based on the key assumption that
the gas consists of point-like particles, which interact only through elastic
collisions, with no long-range intermolecular forces. The main advantage of this
simplified model is its straightforward and easy-to-use EOS, which is described
below:

\begin{equation}
	P = \frac{RT}{V}
\end{equation}

Where $P$ represents the gas pressure, $V$ is the molar volume, $T$ is the
temperature, and $R$ is the ideal gas constant. This equation can also be
expressed in a specific form as follows:

\begin{equation}
	P = \frac{rT}{v} = \rho rT
\end{equation}

Where $\rho = \frac{1}{v}=\frac{M}{V}$ represents the density, $M$ is the fluid's molecular weight, , $v$ is the specific volume of the fluid, and $r$ is known as the specific gas constant:

\begin{equation}
	r = \frac{R}{M}
\end{equation}

	\subsection{Heat capacities, enthalpy and entropy}

	Using the Ideal Gas Equation of State (IG EOS), the following standard
	relationships apply:
	
	\begin{enumerate}
		\item Mayer's relation:
		\begin{equation}
			C_P - C_V = R
		\end{equation}

		\item Variation in internal energy:
		\begin{equation}
			\mathrm{d}U = C_V\mathrm{d}T
		\end{equation}

		\item Variation in enthalpy:
		\begin{equation}
			\mathrm{d}H = C_P\mathrm{d}T
		\end{equation}

		\item Variation in entropy:
		\begin{equation}
			\mathrm{d}S = \frac{C_P}{T}\mathrm{d}T - R\frac{\mathrm{d}P}{P}
		\end{equation}

	\end{enumerate}

	\subsection{NASA Polynomials}
	
	In its simplest form, the Ideal Gas Equation of State (IG EOS) assumes
	constant heat capacities, leading to linear relationships between enthalpy
	and internal energy with respect to temperature. However, this assumption is
	not valid across a wide range of temperatures. For example, in the field of
	combustion, the NASA polynomial model is commonly used to describe heat
	capacities more accurately. The NASA-7 model, for instance, expresses heat
	capacities in the following form \cite{gardiner1984combustion}:

	\begin{equation}
		\frac{C_p}{R} = a_1 + a_2 + a_3 T^2 + a_4 T^3 + a_5 T^4
	\end{equation}

	\begin{equation}
		\frac{H}{RT} = a_1 + \frac{a_2}{2}T + \frac{a_3}{3}T^2 + \frac{a_4}{4}T^3 + \frac{a_5}{5}T^4 + \frac{a_6}{T}
	\end{equation}

	\begin{equation}
		\frac{S}{R} = a_1 \ln T + a_2 T + \frac{a_3}{2}T^2 + \frac{a_4}{3}T^3 + \frac{a_5}{4}T^4 + a_7
	\end{equation}

	Where $a_i$ are constants that depends on the fluid. The ideal gas equation
	of state (EOS) is a straightforward yet accurate model for describing gas
	behavior at high temperatures and relatively low pressures. However, it
	falls short in representing condensed fluids. Despite this limitation, the
	ideal gas model has been extensively studied and documented over the years.
	Consequently, many thermodynamic problems transition from the Real Gas (RG)
	state to the Ideal Gas (IG) state using residual or departure functions (see
	Section 2.4.3.2) to leverage the wealth of existing knowledge associated
	with the IG model.


	\subsection{Compressibility Factor}

	Real fluids at low pressure and high temperature can be accurately modeled by
	the ideal gas equation of state (EOS). However, at lower temperatures or
	higher pressures, a real fluid deviates significantly from ideal gas
	behavior, especially during phase changes, such as when it condenses from a
	gas to a liquid or deposits from a gas to a solid. This deviation is
	quantified by the compressibility factor, $Z$, which indicates how much the
	fluid's behavior differs from that of an ideal gas:

	\begin{equation}
		Z = \frac{PV}{RT} =\frac{Pv}{rT} = \frac{P}{\rho RT} 
	\end{equation}

	Although this definition may appear simple, it serves as a fundamental
	starting point for the development of more complex EOS, such as cubic EOS.
	Before introducing cubic EOS and discussing their suitability for modeling
	supercritical fluids, it is useful to first present some basic EOS. These
	simpler models highlight the limitations of the ideal EOS and illustrate the
	need for more advanced formulations.  

\section{Nobel-Abel and Stiffned Gas Models}

One of the main limitations of the ideal gas model is its inability to account
for intermolecular interactions, particularly those at a distance. To address
this, two of the simplest equations of state (EOS) that include both repulsive
and attractive forces between molecules are the Noble-Abel (NA) and
Stiffened-Gas (SG) models:	

\begin{equation}
	P = \frac{RT}{V - b} 
\end{equation}

\begin{equation}
	P = \frac{RT}{V} - P_{\infty}
\end{equation}

In the Noble-Abel (NA) equation of state (EOS), the repulsive force between
molecules is modeled by the introduction of a covolume $b$, which intuitively
represents the volume that cannot be occupied by the compressed hard spheres at
high pressures in the ideal gas model. In the Stiffened-Gas (SG) EOS, the
attractive forces are represented by a constant pressure $P_{\infty}$, which allows the
consideration of denser phases using a simple EOS. While both models remain
approximations, they accurately capture the asymptotic behavior of a fluid at
low or high temperatures and pressures. 

These two models were later combined in the NASG EOS \cite{le2016noble} to model
two-phase flows, and further extended by incorporating NASA polynomials in
\cite{boivin2019thermodynamic}:

\begin{equation}
	P = \frac{RT}{V - b} - P_{\infty}
\end{equation}

\section{Supercritical Fluids}

Supercritical phenomena are crucial in various industrial applications
\cite{jofre2021transcritical}, particularly in power cycles and energy recovery,
which are the focus of this work. This section is dedicated to understanding the
key differences between supercritical and subcritical fluids, with an emphasis
on their role in power generation.

Supercritical fluids, especially supercritical carbon dioxide ($\ce{CO2}$) and water,
are increasingly important in the development of advanced power cycles due to
their exceptional thermodynamic properties. These fluids are used to enhance
efficiency, reduce the size of equipment, and minimize environmental impact in
power generation systems. Here are some key applications:

    Supercritical $\ce{CO2}$ Power Cycles: Brayton Cycle: Supercritical $\ce{CO2}$ is a
        promising working fluid in closed-loop Brayton cycles, where it operates
        at temperatures and pressures above its critical point. These cycles
        have the potential to surpass the efficiency of traditional steam
        Rankine cycles, primarily because supercritical $\ce{CO2}$ has a higher
        density, which allows for more compact and efficient turbomachinery.
        This increased efficiency translates into lower fuel consumption and
        reduced greenhouse gas emissions, making supercritical $\ce{CO2}$ Brayton
        cycles a leading candidate for next-generation power plants. Waste Heat
        Recovery: One of the significant advantages of supercritical $\ce{CO2}$ cycles
        is their ability to recover waste heat from industrial processes, gas
        turbines, and even nuclear reactors. By utilizing waste heat, these
        systems can generate additional electricity without burning extra fuel,
        improving the overall efficiency of energy systems and reducing the
        carbon footprint of power generation.

    Supercritical Water Power Cycles: Supercritical Water-Cooled Reactors
        (SCWRs): In nuclear power generation, supercritical water is used as a
        coolant in advanced reactor designs. SCWRs operate at supercritical
        pressures and temperatures, allowing for higher thermal efficiency
        compared to conventional reactors. This not only improves the power
        output but also contributes to the safety and economic viability of
        nuclear energy.

    Supercritical Steam Cycles: Traditional fossil fuel power plants are
        evolving with the adoption of supercritical and ultra-supercritical
        steam cycles. By operating at temperatures and pressures above the
        critical point of water, these power plants achieve higher thermal
        efficiencies, reducing fuel consumption and emissions. The move towards
        supercritical steam cycles represents a significant advancement in
        making coal-fired power generation cleaner and more efficient.
		
    Hybrid Power Cycles: Integration with Renewable Energy: Supercritical $\ce{CO2}$
        and water cycles are being integrated with renewable energy sources,
        such as concentrated solar power (CSP) and geothermal energy. In CSP
        plants, supercritical $\ce{CO2}$ is used to transfer heat from the solar field
        to the power block, allowing for more efficient energy conversion.
        Similarly, in geothermal systems, supercritical fluids can enhance
        energy extraction from deep, high-temperature reservoirs, providing a
        reliable and sustainable source of electricity. Combined Cycle Plants:
        In combined cycle power plants, supercritical $\ce{CO2}$ can be used in
        conjunction with gas turbines to further increase overall efficiency.
        The exhaust heat from the gas turbine is used to generate additional
        power in a supercritical $\ce{CO2}$ cycle, making these plants among the most
        efficient forms of fossil fuel-based power generation.

    Environmental Benefits and Future Potential: Carbon Capture and Utilization:
        Supercritical $\ce{CO2}$ power cycles not only improve efficiency but also
        facilitate carbon capture and utilization (CCU). By integrating CCU with
        supercritical $\ce{CO2}$ cycles, power plants can reduce their carbon emissions
        while potentially producing valuable products like synthetic fuels. This
        dual benefit supports the transition to more sustainable energy systems.
        Future Research and Development: Ongoing research into supercritical
        fluid power cycles is focused on optimizing system design, improving
        materials that can withstand the harsh conditions of supercritical
        operation, and integrating these cycles with emerging technologies. The
        goal is to create highly efficient, low-emission power plants that can
        meet the growing global demand for electricity while minimizing
        environmental impact.

In conclusion, supercritical fluids are at the forefront of power generation
technology, offering significant advantages in efficiency, compactness, and
environmental sustainability. As research and development continue, these
systems are expected to play a central role in the future of energy, making them
a critical area of study for advancing power cycles and reducing the carbon
footprint of electricity generation.

In order to star with a better comprehension of the supercritical state
comprehension, some reduced parameters are introduced:

\begin{equation}
	Pr = \frac{P}{P_c}
\end{equation}

\begin{equation}
	T_r = \frac{T}{T_c}
\end{equation}

\begin{equation}
	V_r = \frac{V}{V_c}
\end{equation}

\begin{equation}
	\rho_r = \frac{\rho}{\rho_c}
\end{equation}

where subscript $c$ denotes the parameter's value at the critical point.

	\subsection{Physical Structure of Supercritical Fluids}

	The first topic to address is the nature of supercritical fluids. It is
	well-known that, in the supercritical state, there is no distinctly separate
	phase (see Figure ~\ref{Banuti fluid state plane}) \cite{mcmillan2010going}.

	Using molecular dynamics simulations of liquid and transcritical states at a
	reduced temperature of 0.5 and reduced pressures of 0.7, 1.4, and 3.0,
	Banuti ~\cite{banuti2017seven} demonstrated that there is no need to
	separate quadrants IL and IV in Figure ~\ref{Banuti fluid state plane}. This
	is because densely packed fluids in IL exhibit the same physical behavior as
	their supercritical counterparts in IV. For gases, however, the equivalency
	between the supercritical fluid and the ideal gas model in the ($P_r,T_r$)
	plane is reduced to a line. It is important to note that, up to $P_r=3$ and
	for $T_r>2$, describing a supercritical fluid as an ideal gas results in a
	compressibility factor error of less than 5$\%$, as shown in Figure
	~\ref{BanutiRealGasCompresibility}.

	Consequently, a pseudo-transition should exist, dividing the supercritical
	region, as illustrated in Figure ~\ref{Banuti fluid state plane}. Recent
	independent studies by Gorelli et al. \cite{gorelli2006liquidlike} and
	Simeoni et al. \cite{simeoni2010widom}, using X-ray scattering, have shown
	that sound dispersion, typically observed only in liquids, indeed occurs in
	a supercritical transition. This suggests that there are two distinct
	pseudo-states within supercritical matter, implying the existence of a
	mechanism analogous to a subcritical phase transition.

	\begin{figure}[h!]
		\centering
		\includegraphics[width=0.5\textwidth]{BanutiFluidStatePlane.png}
		\caption{Fluid state plane and supercritical states structure with Widom
		Line (dashed), dividing the supercritical quadrant into a liquid-like
		(subscript LL) and a gas-like (subscript GL) region, extracted from
		\cite{banuti2017seven}.}
		\label{Banuti fluid state plane}
	\end{figure}


	\begin{figure}[h!]
		\centering
		\includegraphics[width=0.5\textwidth]{BanutiRealGasCompresibility.png}
		\caption{Z (solid lines) in pure fluid P r - Tr diagram. Regions of less
		than 5$\%$
		deviation from ideal gas behavior are shaded. The critical point is
		marked by the red circle. Data from NIST
		\cite{burgess2012thermochemical} for nitrogen. Extracted from
		\cite{banuti2017seven}.}
    \label{BanutiRealGasCompresibility}
	\end{figure}

	\subsection{Pseudo-Boiling} % entre [] pour le texte dans la TOC
		\subsubsection{The Widom line}
		The existence of this pseudo-transition is confirmed at loci where the
		isobaric heat capacity ($C_p$) exhibits a relative maximum, either at
		constant pressure or constant temperature, in the ($P_r,T_r$) plane.
		These two definitions correspond to the Widom Line and the Characteristic
		Isobaric Inflection Curve (CIIC), respectively. Recently, Lamorgese et al.
		compiled classical cubic Equation of State (EOS) results to map the Widom
		Line locus \cite{lamorgese2018widom}.

		Banuti, however, advanced this understanding further. By applying the
		theoretical definition of an equilibrium state—where Gibbs free energy
		is equal—and using a correlation from Waring \cite{waring1954form} along
		with a Soave-Redlich-Kwong-based calculation of the subcritical
		coexistence line's slope, he proposed a similarity law that aligns both
		the coexistence and Widom lines across a range of fluids
		\cite{banuti2017similarity}. The strength of Banuti's approach lies in
		its solid theoretical foundation, its adaptability to various fluids,
		and its simplicity.

		\begin{equation}
			P_r = exp\left[\frac{A_s}{min(T_r,1)}(T_r-1)\right]
			\label{ReducePressureEquation}
		\end{equation}

		where $A_s$ is the critical slope of the coexistence line and can be
		calculated using a cubic Equation of State. For example, applying
		the Soave-Redlich-Kwong EOS results in the following expression:

		\begin{equation}
			A_{SRK} = 5.51934 + 4.80640\omega - 0.537437\omega^2
		\end{equation}
		
		where $\omega$ is the acentric factor of the considered species
		\cite{banuti2017similarity}. The corresponding scaling parameter is
		defined as:

		\begin{equation}
			P_r^* = P_r^{\frac{A_0}{A_s}}
		\end{equation}

		where $A_0$ is the result of Eq. ~\ref{ReducePressureEquation} for a
		zero acentric factor. This scaling parameter is illustrated graphically
		in Figure ~\ref{BanutiCollapseWidomLines} below.

		\begin{figure}[h!]
			\centering
			\includegraphics[width=0.5\textwidth]{BanutiCollapseWidomLines.png}
			\caption{Collapse of Widom lines from NIST
			\cite{burgess2012thermochemical} data when using $Pr^*$ instead of P
			r . Extracted from \cite{banuti2017similarity}}
		\label{BanutiCollapseWidomLines}
		\end{figure}

	\subsection{Supercritical Latent Heat}

	To highlight the differences between subcritical and supercritical phase
	transitions, Banuti \cite{banuti2015crossing} illustrated the contrast
	between low-pressure and high-pressure state transitions. As shown in Figure
	~\ref{BanutiLatentHeat}, at low pressures, the latent heat $\Delta h_{th}$
	required to overcome intermolecular forces is gradually replaced by a
	thermal shielding effect $\Delta h_{th}$ at higher pressures. This effect
	forces the fluid to heat up before it can undergo the transition.

	\begin{figure}[h!]
		\centering
		\includegraphics[width=0.8\textwidth]{BanutiLatentHeat.png}
		\caption{Isobaric specific heat capacity in the vicinity of the liquid
		to gas transi- tion at low and high subcritical and at supercritical
		pressures. Extracted from \cite{banuti2019latent}}
	\label{BanutiLatentHeat}
	\end{figure}

	Furthermore, the pseudo-boiling phenomenon involves both dynamic and
	thermodynamic state transitions \cite{banuti2020between,banuti2019latent}.
	Banuti et al. note in their study that at very high pressures, this
	transition loses its thermodynamic nature and is replaced by a dynamic shift
	from rigid to non-rigid liquids, which no longer display liquid-like
	dispersion behavior.

	Section 2.3 highlights the need for an accurate thermodynamic description of
	heat capacities. Models of transport properties, such as thermal conductivity,
	are discussed in Chapter 3 to ensure a proper understanding of supercritical
	(SC) behavior.

			
\section{Cubic EoS}

Once the importance of accurate thermodynamic representations for calculations
involving supercritical fluids has been established, attention can be directed
to more advanced EOS.  

In this section, several of the most widely used cubic equations of state (CEOS)
are introduced. The widespread application of these equations is attributed to
three main factors \cite{tosun2021thermodynamics,poling2001properties}:  

\begin{enumerate}
    \item They are applicable across a broad range of pressures and
    temperatures.  
    \item They are capable of describing both liquid and vapor phases.  
    \item They are particularly suitable for the study of supercritical fluids.  
\end{enumerate}

All thermodynamic properties presented in this chapter are either obtained from
the NIST Chemistry WebBook \cite{burgess2012thermochemical} or calculated using
an EOS.  

Before presenting the CEOS in detail, a brief historical overview is provided to
highlight how these equations have evolved to represent the behavior of
supercritical fluids.  

		
\subsection{A Brief Historical Perspective}

This section provides a short historical overview of the EOS considered in this
chapter. The aim is to illustrate how the fundamental principles behind these
models were established and how they shaped the development of CEOS for
describing supercritical fluids.  

\subsubsection{The Corresponding States Principle and Criticality Conditions}
The Corresponding States Principle (CSP) was first introduced by van der Waals
with the statement: ``Substances behave alike at the same reduced states.
Substances at the same reduced states are in corresponding states.'' This
principle is of fundamental importance for the study of supercritical fluids, as
it introduces the critical point of a fluid (denoted by the subscript $c$) as a
scaling factor. The critical point enables comparisons between different
compounds through the use of reduced thermodynamic variables, which remain
widely applied today.  

The critical point also provides a natural reference for scaling because it is
uniquely defined by specific \emph{criticality conditions}, given by:  

\begin{equation}
  \left(\frac{\partial P}{\partial V}\right)_{T=T_c} = 0
\end{equation}

\begin{equation}
  \left(\frac{\partial^2 P}{\partial V^2}\right)_{T=T_c} = 0
\end{equation}

These conditions reflect the fact that the saturation curve terminates at the
critical point, where the critical isotherm exhibits an inflection. Importantly,
the attractive and repulsive parameters in cubic equations of state are derived
directly from these criticality conditions. As a result, the CSP provides a
framework in which different compounds can be consistently compared.  


	\subsubsection{Van der Waals EOS}
	Historically, CEOS have been expressed in pressure-explicit form. They were
	specifically designed to account for intermolecular forces and accurately
	depict multiphase behavior. Van der Waals was the first to address this
	challenge, introducing the first cubic EOS, for which he was awarded the
	Nobel Prize in 1910 \cite{van1910equation}. His straightforward formulation
	utilizes two constant parameters, a and b, to model attractive and repulsive
	forces, respectively

	\begin{equation}
		P = \frac{RT}{V-b} - \frac{a}{V^2}
	\end{equation}

	This formulation was developed into various equations of state (EOS)
	throughout the second half of the 20th century, particularly to refine the
	attractive term and make it more complex and accurate for different
	applications \cite{wei2000equations}.

	\subsubsection{Redlich and Kwong's modification}
	
	In 1949, Redlich and Kwong, building on Van der Waals’ hard-sphere term,
	introduced an improved EOS by incorporating a more detailed temperature
	dependency into the attractive term \cite{redlich1949thermodynamics}. This
	modification allowed their EOS to provide more accurate predictions of fluid
	behavior, particularly for gases near the critical point:

	\begin{equation}
		P = \frac{RT}{V-b} - \frac{a}{\sqrt(T_r)}\frac{1}{V(V+b)}
	\end{equation}

	Its widespread success stemmed from its adaptability, as Chueh and Prausnitz
	demonstrated that it could accurately predict both liquid and vapor
	saturation properties \cite{chueh1967vapor1, chueh1967vapor2}.

	\subsubsection{Acentric factor introdction}

	Despite their success, both of these CEOS models fall short in representing
	the geometric complexity and polar properties of certain molecules. As
	illustrated in Figure ~\ref{AcentricFactorRepresentation}, in complex polar
	compounds,intermolecular forces arise from locations other than the
	molecular centers.

	\begin{figure}[h!]
		\centering
		\includegraphics[width=0.8\textwidth]{AcentricFactorRepresentation.png}
		\caption{Isobaric specific heat capacity in the vicinity of the liquid
		to gas transi- tion at low and high subcritical and at supercritical
		pressures. Extracted from \cite{banuti2019latent}}
	\label{AcentricFactorRepresentation}
	\end{figure}

	To address this issue, Pitzer introduced the acentric factor $\omega$ in
	1955 \cite{pitzer1955volumetric}, as a measure of how much the thermodynamic
	properties of a substance deviate from those predicted by the CSP.

	\subsubsection{Soave's contribution}

	In 1972, Soave, recognizing the significance of Pitzer's acentric factor,
	extended the Redlich-Kwong equation of state (EOS) to create the widely
	known Soave-Redlich-Kwong (SRK) EOS \cite{soave1972equilibrium}:

	\begin{equation}
		P = \frac{RT}{V-b}-\frac{a[1+K_{SRK}(\omega)(1-\sqrt(T_r))]^2}{V(V+b)}
	\end{equation}

	\begin{equation}
		K_{SRK}(\omega) = 0.480 + 1.574\omega -0.176\omega^2
	\end{equation}

	This formulation is particularly well-suited for accurately predicting vapor
	pressure data and behavior in the near-critical region
	\cite{tosun2021thermodynamics}.

	\subsubsection{Peng-Robinson equation of state}

	In 1976, Peng and Robinson (PR) recognized that the SRK EOS overestimated
	the critical compressibility factor of fluids ($Z_{c,SRK} = 1/3$). In response,
	they proposed an alternative form for the attractive term along with a more
	complex volume dependency \cite{peng1976new}:

	\begin{equation}
		P = \frac{RT}{V-b}-\frac{a[1+K_{SRK}(\omega)(1-\sqrt(T_r))]^2}{V^2 +2V -b^2}
	\end{equation}

	\begin{equation}
		K_{SRK}(\omega) = 0.37464 + 1.54226\omega -0.26992\omega^2
	\end{equation}

	This formulation improved the accuracy of liquid volume predictions and
	provided more precise results around the critical point ($Z_{c,PR} = 0.307$).

	\subsection{Derived Properties for a Generic CEOS}
	This section focuses on establishing thermodynamic properties derived from
	CEOS. To provide a foundation for this, we begin with a few key reminders.
	Basic relationships between energy functions are introduced, followed by the
	application of the first law of thermodynamics to present fundamental
	equations that link differential expressions of various thermodynamic
	variables. Maxwell's relations are also recalled.

	Next, we demonstrate that Helmholtz free energy can serve as a generating
	function for other thermodynamic quantities. The concept of residual properties
	is then introduced, and these are applied to CEOS to determine the final
	expressions for important state functions, based on a generic CEOS.

	\subsubsection{Few reminders}

	Basic Definitions: In this section, we revisit the definitions of key
	thermodynamic quantities, including specific enthalpy $h$, Helmholtz free
	energy $A_W$, Gibbs free energy $g$, and the compressibility factor $Z$.

	\begin{equation}
		h = e + P/\rho
	\end{equation}

	\begin{equation}
		A_W = e + Ts
	\end{equation}

	\begin{equation}
		g = h - Ts
	\end{equation}

	\begin{equation}
		Z = \frac{P}{\rho RT}
	\end{equation}

	Fundamental equations: At this point, we need to establish relationships
	between the energetic variables and the PVT (Pressure-Volume-Temperature)
	state of a system. The first law of thermodynamics provides the fundamental
	equation for internal energy e (see Eq. ~\ref{ThermodynamicFirstLaw}). By
	combining this with the definitions of specific enthalpy, Helmholtz free
	energy, and Gibbs free energy, we derive the remaining fundamental
	equations. These equations are presented for a single species. Additionally,
	to transition between volume and density formulations, the following
	identity is applied: $dv = -\frac{d\rho}{\rho ^2}$

	\begin{equation}
		de = Tds - Pdv = Tds + \frac{P}{\rho ^2}d\rho
		\label{ThermodynamicFirstLaw}
	\end{equation}

	\begin{equation}
		dh = Tds + vdP = Tds + \frac{1}{\rho}dP
	\end{equation}

	\begin{equation}
		dA_W = -Pdv -sdT = \frac{P}{\rho ^2}d\rho - sdT
	\end{equation}

	\begin{equation}
		dg = vdP - sdT = \frac{1}{\rho}dP -sdT
	\end{equation}

	Maxwell rules: Some mathematical tools required to derive the so-called
	Maxwell relations are presented in Section A.1. The Maxwell relations,
	derived from Equations (2.34) through (2.37), are then given as follows:

	\begin{equation}
		\left(\frac{\partial T}{\partial \rho}\right)_s = \frac{1}{\rho ^2}\left(\frac{\partial P}{\partial s}\right)_{\rho}
	\end{equation}

	\begin{equation}
		\left(\frac{\partial T}{\partial P}\right)_s = \left(\frac{\partial 1/\rho}{\partial s}\right)_P 
	\end{equation}

	\begin{equation}
		\frac{1}{\rho ^2}\left(\frac{\partial P}{\partial T}\right)_{\rho} = -\left(\frac{\partial s}{\partial \rho}\right)_T
	\end{equation}

	\begin{equation}
		\left(\frac{\partial 1/\rho}{\partial T}\right)_P = -\left(\frac{\partial s}{\partial P}\right)_T
	\end{equation}

	These relations will soon be used to demonstrate that Helmholtz free energy
	serves as a generating function for other important thermodynamic variables.

	\subsection{Residual Properties}

	In this section, we introduce residual properties, which serve as transition
	functions between a fluid in a real gas state and its ideal gas equivalent
	at the same thermodynamic conditions ($\rho$,$T$). Here, $\rho$ (density) and $T$
	(temperature) are used to describe the thermodynamic state, as these are the
	independent variables in CEOS formulations.

For a given state function $\phi$, its residual property $\phi^R$ is defined as
follows:

\begin{equation}
	\phi^r(\rho, T) = \phi(\rho, T) - \phi^*(\rho, T) = 
\end{equation}

The key idea here is to utilize residual properties in conjunction with the
generating properties of the Helmholtz energy and a suitable description of the
ideal gas state. For this purpose, a reformulated version of eq. (2.34) is used
to analytically calculate variations of $A_W(\rho,T)$ along a strategically
chosen path. Since $A_W$ is a state function, this path can be freely selected.

A practical approach is to integrate residual quantities along isotherms, given
that ideal gas properties depend solely on temperature.

Furthermore, since ideal gas properties are temperature-dependent and follow
$Z=1$ (the PVT relationship for ideal gases), we simplify the integration by
taking the limit $V=+\inf$ (or equivalently $\rho=0$). This approach eliminates
integration constants that would otherwise cancel out throughout the entire
calculation. The resulting simplifications are illustrated in the following
equation, where redundant constants are explicitly removed through matching
terms:

\[
\Delta_{(T_0, \rho_0) \rightarrow (T, \rho)} \phi = \Delta_{(T_0, \rho_0) \rightarrow (T_0, \rho)} \phi^* + \Delta_{(T_0, \rho_0) \rightarrow (T, \rho)} \phi^* + \Delta_{(T, \rho)^* \rightarrow (T, \rho)} \phi
\]
\[
= \phi^*(T_0) - \phi(T_0, \rho_0) + \phi^*(T) - \phi^*(T_0) + \phi(T, \rho) - \phi^*(T)
\]

The complete integration path for calculating the thermodynamic state function
$\phi$ is illustrated in Figure ~\ref{ResidualPath}.

\begin{figure}[h!]
	\centering
	\includegraphics[width=0.8\textwidth]{ResidualPath.png}
	\caption{Isobaric specific heat capacity in the vicinity of the liquid
	to gas transi- tion at low and high subcritical and at supercritical
	pressures. Extracted from \cite{banuti2019latent}}
\label{ResidualPath}
\end{figure}

First, Section 2.4.3.3 will demonstrate that the residual Helmholtz energy acts
as a generating function for other residual properties of interest. Then, in
Section 2.4.3.4, these resulting expressions will be transformed to relate key
residual properties to the compressibility factor ZZ and its derivatives.
Finally, these expressions will be reformulated in terms of variables compatible
with pressure-explicit equations of state (EOS), yielding temperature- and
density-dependent relationships.

\subsection{Residual Properties from Helmholtz Energy}

In this section, we will use the expressions from Section 2.4.3.1 along with
fundamental definitions to link residual properties to the residual Helmholtz
energy. The goal of this approach is to demonstrate that $A_W^R$ serves as a
generating function for other residual properties. Starting from the definition
of $A_W^R$ and equation (2.34), we derive the total differential of
$\frac{A_W^R}{rT}$. By applying the properties of exact differentials, we reveal the
connections between the residual compressibility factor, residual internal
energy, and the derivatives of $A_W(\rho,T)$.

\begin{equation}
\mathrm{d} \left( \frac{A_W^R}{r T} \right) = \frac{Z^R}{\rho} \, \mathrm{d}\rho - \frac{e^R}{r T^2} \, \mathrm{d}T 
\implies 
\begin{cases}
    Z^R = \rho \left( \frac{\partial}{\partial \rho} \left( \frac{A_W^R}{r T} \right) \right)_{T} \\[10pt]
    \frac{e^R}{r T} = -T \left( \frac{\partial}{\partial T} \left( \frac{A_W^R}{r T} \right) \right)_{\rho}
\end{cases}
\end{equation}

Residual entropy and enthalpy are then derived from the residual internal energy
and Helmholtz energy. This is achieved by applying equations (2.31a) and (2.31b)
to the ideal gas state and subtracting the results from those of the real gas
state.

\begin{equation}
	\frac{s^R}{r} = \frac{e^R}{rT} - \frac{A^R_W}{rT} 
\end{equation}

\begin{equation}
	\frac{h^R}{rT} = \frac{e^R}{rT} + Z^R
\end{equation}

Finally, as before, $Z^R , s^R, h^R$ and $e^R$ can be explicited through $A_W^R$:

\begin{equation}
	Z^R = Z - 1 = \rho\left(\frac{\partial(A_W^R/rT)}{\partial\rho}\right)_T
\end{equation}

\begin{equation}
	\frac{e^R}{rT} = -T \left(\frac{\partial(A_W^R/rT)}{\partial T}\right)_\rho
\end{equation}

\begin{equation}
	\frac{s^R}{r} = -T \left(\frac{\partial(A_W^R/rT)}{\partial T}\right)_\rho - \frac{A^R_W}{rT}
\end{equation}

\begin{equation}
	\frac{h^R}{rT} = -T \left(\frac{\partial(A_W^R/rT)}{\partial T}\right)_\rho + \rho\left(\frac{\partial(A^R_W/rT)}{\partial\rho}\right)_T
\end{equation}

\begin{equation}
	\frac{g^R}{rT} = \rho\left(\frac{\partial(A^R_W/rT)}{\partial\rho}\right)_T + \frac{A^R_W}{rT}
\end{equation}

\subsection{Residual Properties from $Z$ along and Isotherm}

In the previous sections, it was shown that the residual term $\frac{A_W^R}{rT}$ can
serve as a fundamental building block for other residual properties. Now,
integration along an isothermal path (and an iso-compositional path in the case
of mixtures) is performed to obtain density-explicit integrals. Starting from
equation (2.37), equation (2.40a) is derived by choosing this specific
integration path. Differentiating equation (2.40a) at constant density then
yields equation (2.40b).

\begin{equation}
	\frac{A^R_W}{rT} = \int_0^\rho\frac{Z^R}{\rho}d\rho
\end{equation}

\begin{equation}
	\frac{e^R}{rT} = -T\int_0^\rho\left(\frac{\partial Z^R}{\partial T}\right)_\rho\frac{d\rho}{\rho}
\end{equation}

Residual entropy and enthalpy are then explicited using Equations (2.38a) and
(2.38b):

\begin{equation}
	\left(\frac{s^R}{r}\right)_T = -T\int_0^\rho\left(\frac{\partial Z^R}{\partial T}\right)_\rho\frac{d\rho}{\rho} - \int_0^\rho\frac{Z^R}{\rho}d\rho 
\end{equation}

\begin{equation}
	\left(\frac{h^R}{rT}\right)_T = -T\int_0^\rho\left(\frac{\partial Z^R}{\partial T}\right)_\rho\frac{d\rho}{\rho} + Z^R
\end{equation}

\subsection{Application to cubic Equations of State}

From Section 2.4.3.4, only a few key components are needed to derive analytical
expressions for all residual functions. This section aims to compute these
expressions using a cubic equation of state. The pressure-volume-temperature
(PVT) relationship described by this EOS takes the form of equation (2.29),
restated here for reference:

MAKE THAT USING PENG ROBINSON EQUATION OF STATE.

\subsection{Heat Capacities}
Isochoric Heat Capacity: First, a comparison is made to highlight the existence
of a departure function for $c_v$. By definition,
\begin{equation}
c_v = \left( \frac{\partial e}{\partial T} \right)_\rho = c_v^* + \underbrace{\left( \frac{\partial e^R}{\partial T} \right)_\rho}_{\equiv c_v^R}
\end{equation}

Here, $c_v^*$ represents the isochoric heat capacity of the ideal gas, while
$c_v^R$ denotes the residual isochoric heat capacity, which still needs to be
determined. This calculation will be based on Equation (2.45b)."

Isobaric Heat Capacity: 

\subsection{Speed of Sound}
The speed of sound, $c$, is calculated using the following expression from
\cite{riazi1993use}. The partial derivative in this expression is explicitly
detailed in Equation (2.49b):

\begin{equation}
	c^2 = \frac{c_p}{c_v}\left(\frac{\partial p}{\partial \rho}\right)_T
\end{equation}

\section{Modeling of transport Coefficients}
As discussed in Section 2.3.2.2, the pseudo-boiling transition is both a
thermodynamic and dynamic process \cite{banuti2020between}, highlighting the
need for an accurate description of transport properties in pseudo-boiling
scenarios. This chapter focuses on the models used in this study to account for
these phenomena. It is organized into two sections:

Viscosity (shear viscosity) correlations: First, we examine viscosity
correlations, as viscosity plays a key role in the momentum conservation
equation (Equation 1.2) and drives shear stresses (Equation 1.3).

Thermal conductivity: Next, we discuss the thermal transport coefficient, or
thermal conductivity. This parameter influences thermal conduction and is
essential for calculating the heat flux (refer to Equation 1.9), which is
included in the total energy conservation equation (Equation 1.7).

The next subsection refers to the calculation of the viscosity properties. They
are a summary of the paper of ~\cite{}. In his work they developpe a new
viscosity correlation

\subsection{Viscosity}

To determine the viscosity, we use the correlation developed by A. Laesecke and
C. D. Muzny \cite{laesecke2017reference}, as it is specifically designed for
carbon dioxide. The viscosity correlation is defined as:
\begin{equation}
    \eta(T,\rho) = \eta_0(T) + \rho\eta_1(T) + \Delta\eta_r(\rho,T) + \Delta\eta_c(\rho,T)
\end{equation}

In this equation:
\begin{itemize}
    \item $\eta$ represents the dynamic viscosity, measured in mPa·s,
    \item $T$ is the temperature in Kelvin (K), and
    \item $\rho$ is the density in kg/m$^3$.
\end{itemize}

Each term in the expression has a specific role:
\begin{itemize}
    \item $\eta_0(T)$ is the viscosity at zero density,
    \item $\eta_1(T)$ represents the linear-in-density contribution to the viscosity,
    \item $\Delta \eta_r(\rho, T)$ is the residual viscosity, accounting for interactions beyond the ideal gas limit, and
    \item $\Delta \eta_c(\rho, T)$ describes the enhancement of viscosity near the gas-liquid critical point.
\end{itemize}

\subsection{Zero density viscosity}

This contribution reflects the fact that, for ideal gases, thermodynamic
properties depend only on temperature. In this limit, density plays no role and
has no relevance for the contribution.

The physical interpretation is that molecular interactions are limited to binary
collisions, which is somewhat counterintuitive. The mathematical formulation is
based on the intermolecular potential energy surface and is expressed as:

\begin{equation}
    \eta_0(T) = \frac{1.0055\sqrt{T}}{a_0 + a_1T^{\frac{1}{6}}+ a_2\exp{a_3T^{\frac{1}{3}}} + \frac{a_4 + a_5T^{\frac{1}{3}}}{\exp{T^{\frac{1}{3}}}} + a_6\sqrt{T}}
	\label{ZeroDensityEq}
\end{equation}

The coefficients required for the calculations are presented in
Table~\ref{ZeroDensityCoef}. In addition, according to
\cite{laesecke2017reference}, the factor 1.0055 is introduced to improve the
agreement with experimental results.

\begin{table}[h]
    \centering
    \begin{tabular}{c|c}
        $i$ & $a_i$\\
        \hline  
        0 & 1749.354893188350\\
        1 & -369.069300007128\\
        2 & 5423856.34887691\\
        3 & -2.21283852168356\\
        4 & -269503.247933569\\
        5 & 73145.021531826\\
        6 & 5.34368649509278\\
    \end{tabular}
    \caption{Values of the parameters $a_i$ in Eq. ~\ref{ZeroDensityEq}}
    \label{ZeroDensityCoef}
\end{table}

The performance of this equation are shown in
Figs.~\ref{fig:FracDeviationsViscosity}-~\ref{fig:FracDeviationsViscosity3}.
Figure ~\ref{fig:FracDeviationsViscosity} presents the relative deviations
between values obtained from Eq.~ref{ZeroDensityEq} and the unscaled data of
Hellmann ~\cite{hellmann2014ab}. Figure ~\ref{fig:FracDeviationsViscosity2}
illustrates the representation of Hellmann’s unscaled $\eta_0$ data together
with the extrapolation behavior of Eq. ~\ref{ZeroDensityEq} over the temperature
range 0–2100 K. Finally, Fig.~\ref{fig:FracDeviationsViscosity3} compares Eq.
~\ref{ZeroDensityEq} with $\eta_0$ data derived from the most accurate available
measurements in the range 100–1900 K.

\begin{figure}[h!]
	\centering
	\includegraphics[width=0.8\textwidth]{FracDeviationsViscosity.png}
	\caption{Fractional deviations of viscosity data for \ce{CO2} in the limit
	of zero density calculated with the new correlation Eq. ~\ref{ZeroDensityEq}
	from the PES-based data calculated by
	Hellmann~\cite{hellmann2014ab}.Extracted from ~\cite{laesecke2017reference}}
\label{fig:FracDeviationsViscosity}
\end{figure}

\begin{figure}[h!]
	\centering
	\includegraphics[width=0.8\textwidth]{FracDeviationsViscosity2.png}
	\caption{Representation of the viscosity of \ce{CO2} in the limit of zero
	density by Eq. ~\ref{ZeroDensityEq} compared to the unscaled data of
	Hellmann~\cite{hellmann2014ab} calculated to 100 K and extrapolation behavior of
	Eq.~\ref{ZeroDensityEq} in the limit T / 0 K. Note that for this comparison
	the factor 1.0055 in the numerator of Eq.~\ref{ZeroDensityEq} was set to
	unity. Extracted from \cite{laesecke2017reference}}
\label{fig:FracDeviationsViscosity2}
\end{figure}

\begin{figure}[h!]
	\centering
	\includegraphics[width=0.8\textwidth]{FracDeviationsViscosity3.png}
	\caption{Comparison of $\eta_0$ data derived from measurements considered to
	be most accurate and those calculated by Bock et
	al.~\cite{bock2002calculation} based on the PES of Bukowski et
	al.~\cite{bukowski1999intermolecular} relative to the new correlation
	Eq.~\ref{ZeroDensityEq} in the temperature range 100 to 1900 K. Extracted
	from \cite{huber2016reference}}
\label{fig:FracDeviationsViscosity3}
\end{figure}


\subsection{Initial density dependence of viscosity}

This part represent the first variations where density starts to have a little
impact on the calculation of the viscosity. In this regime this dependency
has a linear shape.

\begin{equation}
    \eta_1(T) = \eta_0(T)B^*_\eta(T^*)\sigma^3N_A/M 
	\label{InitialDensityViscoEq}
\end{equation}

In this equation $\eta_0$ represent the zero density viscocity. $\sigma$ is the
length scaling parameter, that has the value of 0.378421 nm (The details of the
computation could be found in ~\cite{laesecke2017reference}). $N_A$ is the
Avogadro constant and is equal to $6.022140857(74)\times 10^{23}
\mathrm{mol}^{-1}$. The molar mass of the $\ce{CO2}$ represented by $M = 44.0095
\times 10^{-3} \mathrm{kg mol^{-1}}$. And finally, to compute the second
viscosity virial coefficient $B^*_\eta(T^*)$ the equation
$~\ref{SecondVirialCoef}$ was used.  

\begin{equation}
    B^*_\eta(T^*)=b_0+\sum^8_{i=1}\frac{b_i}{(T^*)^{t_i}}
	\label{SecondVirialCoef}
\end{equation}

All the necesary parameters to make the calculations in
eq.$~\ref{SecondVirialCoef}$ are showed in $~\ref{tab:SecondVirialCoef}$.

\begin{table}[ht]
    \centering
    \begin{tabular}{c|c|c}
    $i$&$b_i$&$t_i$\\
    \hline
    0   & -19.572881    &—\\  
    1   & 219.73999     &0.25\\       
    2   & -1015.322 6   &0.5 \\   
    3   & 2471.012 5    &0.75\\   
    4   & -3375.171 7   &1  \\
    5   & 2491.659 7    &1.25\\   
    6   & -787.260 86   &1.5 \\   
    7   & 14.085 455    &2.5 \\   
    8   & -0.346 641 58 &5.5\\
    \end{tabular}
    \caption{Values of the parameters $b_i$ and exponents $t_i$ in
    Eq.~\ref{SecondVirialCoef}}
    \label{tab:SecondVirialCoef}
\end{table}

\begin{table}[ht]
    \centering
    \begin{tabular}{c c}
    \hline
    \hline
    $\gamma$   & 8.062 827 374 812 77    \\  
    $c_1$ & 0.360 603 235 428 487    \\       
    $c_2$   & 0.121 550 806 591 497   \\
    \hline
    \end{tabular}
    \caption{Values of the parameters in Eq.~\ref{ResidualViscosityEq}}
    \label{tab:my_label3}
\end{table}

Equation~\ref{InitialDensityViscoEq} is derived from
data~\cite{vogel1998reference} in the range $0.5 \leq T^{*} \leq 100$, but can
be safely extrapolated down to $T^{*} = 0.3$ and beyond $T^{*} = 100$.
Application of this dimensionless function to real compounds requires knowledge
of the energy scaling parameter $\varepsilon / k_{B}$, which provides the
conversion between the absolute temperature $T$ and the reduced temperature
$T^{*} = k_{B} T / \varepsilon$.


\subsection{Residual viscosity}

\begin{equation}
    \Delta\eta_r(T,\rho) = \eta_{tL}(c_1T_r\rho_r^3 + (\rho_r^2 + \rho_r^\gamma)/(T_r -c_2))
	\label{ResidualViscosityEq}
\end{equation}

\begin{equation}
    \eta_{tL} = \frac{\rho_{tL}^{2/3} \sqrt{RT_t}}{M^{1/6}N_A^{1/3}}
\end{equation}

\subsection{Critical enhancement viscosity}
This viscosity is really small and is only important near to the critical condition (For $\ce{CO2}$ from 304.29$\leq T\leq$304.99 K. Because this value is really small it's going to be neglected in the predictions of the total viscosity (for more information read the paper). 
\begin{equation}
    \Delta\eta_c = \eta_b(T,\rho)\exp{\left(z_\eta H\right)}
\end{equation}


\begin{figure}[h!]
	\centering
	\includegraphics[width=0.8\textwidth]{TotalErrorViscoCO2.png}
	\caption{. Extracted
	from \cite{laesecke2017reference}}
\label{fig:TotalErrorViscoCO2}
\end{figure}


\section{Conductivity}
In order to determine the conductivity, the paper in titled Reference
Correlation of the Thermal Conductivity of Carbon Dioxide from the Triple Point
to 1100 K and up to 200 MPa was used ~\cite{}.

According to this paper, the conductivity could be calculated as:
\begin{equation}
    \lambda(T,\rho) = \lambda_0(T) + \Delta\lambda(T,\rho) + \Delta\lambda_c(T,\rho)    
\end{equation}

Where three independent components are used to modeling the total thermal
conductvity. As in the correlations for the viscocity, the first term
$\lambda_0(T) = \lambda(0,T)$ is related with the  thermal conductivity in the
dilute-gas limit, where only twi-body molecular interactions occur. The second
term $\Delta\lambda(T,\rho)$, called also the residual property, represents the
contribution of all other effects to the thermal conductivity of the fluid at
elevated densities, including many-body collisions, molecular-velocity
correlations, and collisional transfer. The last term,
$\Delta\lambda_c(T,\rho)$,called also critical enhancement thermal conductivity,
arises from the long-range density fluctuations that occur in a fluid near its
critical point, which contribute to divergence of the thermal conductivity at
the critical point. 

\subsection{Dilute gas conductivity}

In order to determine the correlation that gives the values of the Dilute gas
conductivity, the data used by ~\cite{huber2016reference} contained 69 data
point from 150 to 2000 K.
\begin{equation}
    \lambda_0(T) = \frac{\sqrt{T_r}}{\sum^3_{k=0}\frac{L_k}{T_r^k}}
	\label{eq:DiluteGasConductivity}
\end{equation}

The coefficients used in this formula are given in
Tab.~\ref{tab:DiluteGasConductivity}.

\begin{table}[h]
    \centering
    \begin{tabular}{c|c}
         $k$ & $L_k$  \\
         \hline
         0  &   $1.51874307\times10^{-2}$\\
         1  &   $2.80674040\times10^{-2}$\\
         2  &   $2.28564190\times10^{-2}$\\
         3  &   $-7.41624210\times10^{-3}$\\
    \end{tabular}
    \caption{Caption}
    \label{tab:DiluteGasConductivity}
\end{table}

As showed in Fig.\ref{fig:DiluteGasConductivity} good agreement is probed by
using the equation ~\ref{eq:DiluteGasConductivity}.

\begin{figure}[h!]
	\centering
	\includegraphics[width=0.8\textwidth]{LambdaZero.png}
	\caption{Isobaric specific heat capacity in the vicinity of the liquid
	to gas transi- tion at low and high subcritical and at supercritical
	pressures. Extracted from \cite{huber2016reference}}
\label{fig:DiluteGasConductivity}
\end{figure}

\subsection{Residual conductivity}

The thermal conductivities of pure fluids have an enhancement around the
critical point and have an infinite value at the criticical point. The procedure
use in ~\cite{huber2016reference}. The procedure that Huber et al. uses in their
work use the ODRPACK ~cite{boggs1992user} to fit the primary data to determine
the coefficients in Eq.~\ref{eq:ResidualConductivity}. What is important to note
is that during the regression process, they found that coefficients $B_{2,i}$
were not important for the representation of supercritical and vaport phase data.
This observation is also reported by Vesovic et al.~\cite{vesovic1990transport}.  

\begin{equation}
    \Delta\lambda(\rho,T) = \sum_{i=1}^6(B_{1,i} + B_{2,i}(T/T_c))(\rho/\rho_c)^i
	\label{eq:ResidualConductivity}
\end{equation}

The values of this coefficients coud be found in
Tab.~\ref{tab:ResidualConductivity}. 

\begin{table}[h]
    \centering
    \begin{tabular}{c|c|c}
         $i$ & $B_{1,i}(Wm^{-1}k^{-1})$  & $B_{2,i}(Wm^{-1}k^{-1})$\\ 
         \hline
         1  &   $1.00128\times10^{-2}$   & $4.3089\times10^{-3}$\\
         2  &   $5.60488\times10^{-2}$   & $-3.58563\times10^{-2}$\\
         3  &   $-8.11620\times10^{-2}$  & $6.71480\times10^{-2}$\\
         4  &   $6.24337\times10^{-2}$   & $-5.22855\times10^{-2}$\\
         5  &   $-2.06336\times10^{-2}$  & $1.74571\times10^{-2}$\\
         6  &   $2.53248\times10^{-3}$   & $1.96414\times10^{-3}$\\       
    \end{tabular}
    \caption{Coefficients of Eq.~\ref{eq:ResidualConductivity} for the residual
	thermal conductivity of carbon dioxide}
    \label{tab:ResidualConductivity}
\end{table}

\subsection{Critical enhancement conductivity}

In the original paper of Huber et al. ~\cite{huber2016reference}, they proposed
to methods to calculate the critical enhancement conductivity. 

The first method used the model proposed in ~\cite{olchowy1989simplified}, where
precise calculations of $C_p$ and $C_v$ are needed. The model is given by:

\begin{equation}
	\Delta \lambda_c = \frac{\rho C_p R_D k_B T}{6 \pi \bar{\eta}\xi}(\bar{\Omega} - \bar{\Omega}_0)
\end{equation}

with

\begin{equation}
	\bar{\Omega} = \frac{2}{\pi}\left[ \left( \frac{C_p - C_v}{C_p}\right) \mathrm{arctan}(\bar{q}_D\xi) + \frac{C_v}{C_p}\bar{q}_D\xi \right]
\end{equation}

and

\begin{equation}
	\bar{\Omega}_0 = \frac{2}{\pi}\left[ 1 - \mathrm{exp}\left( - \frac{1}{(\bar{q}_D\xi)^{-1} + (\bar{q}_D\xi\rho_c/\rho)^2/3}\right)\right]
\end{equation}

The correlation length, $\xi$, is given by:

\begin{equation}
	\xi = \xi_0\left(\frac{p_c\rho}{\Gamma\rho_c^2}\right)^{\nu/\gamma}
	\left[ \left. \frac{\partial \rho(T,p)}{\partial p} \right|_{T} - \left(\frac{T_{ref}}{T}\right) \left.\frac{\partial \rho(T_{ref},\rho)}{\partial p}\right|_{T}\right]^{\nu/\gamma}
\end{equation}

This model requires some universal constants ~\cite{perkins2013simplified} like
,$R_D = 1.02$, $\nu =0.63$ and $\gamma = 1.239$ and some system-dependent
amplituds like $\Gamma = 0.052$ and $\xi_0 = 1.50 \times 10^{-10}$ m. The
reference temperature is given by $T_{ref} = (3/2)T_c$. The cutoff wavelength
$\bar{q}_D^{-1}$ is equal to $4.0 \times 10^{-10}$ m. 

As discussed in the introduction chapter, in this manuscript, the author is
interested in the supercritical state. Figure
~\ref{fig:ThermalConductivityError} shows the relative error of this formulation
for the supercritical state for the \ce{CO2} as function of temperature uo to
500 K (denisities from 0.02 to 341.81 $\mathrm{kg m^{-3}}$, pressures from
0.0001 to 7.2 MPa).We can see that the bigest error appears near the critical
point, and has as maximum discrepency with experimental data of 8\%.

\begin{figure}[h!]
	\centering
	\includegraphics[width=0.8\textwidth]{ThermalConductivityError.png}
	\caption{Percentage deviations of primary experimental data of carbon dioxide
	from the values calculated by the present model as a function of
	temperature, for the supercritical region at temperatures to 500 K.
	Extracted from \cite{huber2016reference}}
\label{fig:ThermalConductivityError}
\end{figure}


Another simpler formulation was developped in the work of
~\cite{huber2016reference}, where when state points are 10 K more than the
critical point, the value of the critical enhancement thermal conductivity could
be represented within about 5\% error by the next empirical expression:

\begin{equation}
    \Delta\lambda_c(\rho,T) = \frac{-17.47-44.88\Delta T_c}{0.8563-exp[8.865\Delta T_c +4.16\Delta \rho_c^2 + 2.302\Delta T_c \Delta\rho_c - \Delta\rho_c^3] -0.4503\Delta\rho_c - 7.197\Delta T_c}    
\end{equation}

where $\Delta T_c= \frac{T}{T_c}-1$ and $\Delta \rho_c = \frac{\rho}{\rho_c}-1$.
Unlike the model of Olchowy and Sengers ~\cite{olchowy1989simplified}, which
requires precise data on the compressibility, specific heat, and viscosity of
carbon dioxide in the critical region, this equation does not rely on such
detailed thermophysical properties. Nevertheless, it lacks a rigorous
theoretical foundation and does not converge to the correct asymptotic behavior
at the critical point.

\begin{figure}[h!]
	\centering
	\includegraphics[width=0.8\textwidth]{TotalErrorCondCO2.png}
	\caption{. Extracted
	from \cite{laesecke2017reference}}
\label{fig:TotalErrorCondCO2}
\end{figure}

\subsection{Thermal Conductivity}

\begin{figure*}[htbp]
    \centering
    % First row
    \subfloat[]{%
        \includesvg[width=0.45\textwidth]{InternalEnergyVSTemp(1)}%
        \label{fig:subfig1}}%
    \hfill
    \subfloat[]{%
        \includesvg[width=0.45\textwidth]{SoundSpeedVSTemp(1)}%
        \label{fig:subfig2}}%

    % Second row
    \subfloat[]{%
        \includesvg[width=0.45\textwidth]{CpVSTemp(1)}%
        \label{fig:subfig3}}%
    \hfill
    \subfloat[]{%
        \includesvg[width=0.45\textwidth]{ZVSTemp(1)}%
        \label{fig:subfig4}}%

    % Third row
    \subfloat[]{%
        \includesvg[width=0.45\textwidth]{RhoVSTemp(1)}%
        \label{fig:subfig5}}%
    \hfill
    \subfloat[]{%
        \includesvg[width=0.45\textwidth]{betaVSTemp(1)}%
        \label{fig:subfig6}}%

    \caption{
    Thermodynamic properties of the s\ce{CO2} computed using the Peng–Robinson equation of state as functions of temperature. 
    (a) Internal energy, 
    (b) speed of sound, 
    (c) isobaric heat capacity \(C_p\), 
    (d) compressibility factor \(Z\), 
    (e) density \(\rho\), and 
    (f) thermal expansion coefficient \(\beta\). 
    All properties are evaluated at a constant pressure of 80 bar.
    }
    \label{fig:multi1}
\end{figure*}


\begin{figure*}[htb]
    \centering
    % First row
    \subfloat[]{%
        \includegraphics[width=0.45\textwidth]{lambdaVStempN2.pdf}%
        \label{fig:subfig7}}%
    \hfill
    \subfloat[]{%
        \includegraphics[width=0.45\textwidth]{viscoVStempN2.pdf}%
        \label{fig:subfig8}}%

    % Second row
    \subfloat[]{%
        \includegraphics[width=0.45\textwidth]{lambdaVStempCO2.pdf}%
        \label{fig:subfig9}}%
    \hfill
    \subfloat[]{%
        \includegraphics[width=0.45\textwidth]{viscoVStempCO2.pdf}%
        \label{fig:subfig10}}%

    \caption{Comparison of thermal conductivity (left) and viscosity (right) for \ce{N2} and \ce{CO2} as functions of temperature at \(P = 39.7\) bar (top) and \(80\) bar (bottom).}
    \label{fig:multi2}
\end{figure*}
