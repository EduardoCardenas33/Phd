\chapter{ Thermodynamics}
\chaptertoc{}

\section{Ideal Gas EoS}

A mathematical relationship that connects temperature, pressure, and volume is
known as an Equation of State (EOS). These equations can be expressed in either
a pressure-explicit form (such as CEOS) or a volume-explicit form. The ideal gas
(IG) model is the most basic theoretical representation of a gas, grounded in
the principles of kinetic theory. This model is based on the key assumption that
the gas consists of point-like particles, which interact only through elastic
collisions, with no long-range intermolecular forces. The main advantage of this
simplified model is its straightforward and easy-to-use EOS, which is described
below:

\begin{equation}
	P = \frac{RT}{V}
\end{equation}

Where $P$ represents the gas pressure, $V$ is the molar volume, $T$ is the
temperature, and $R$ is the ideal gas constant. This equation can also be
expressed in a specific form as follows:

\begin{equation}
	P = \frac{rT}{v} = \rho rT
\end{equation}

Where $\rho = \frac{1}{v}=\frac{M}{V}$ represents the density, $M$ is the fluid's molecular weight, , $v$ is the specific volume of the fluid, and $r$ is known as the specific gas constant:

\begin{equation}
	r = \frac{R}{M}
\end{equation}

	\subsection{Heat capacities, enthalpy and entropy}

	Using the Ideal Gas Equation of State (IG EOS), the following standard
	relationships apply:
	
	\begin{enumerate}
		\item Mayer's relation:
		\begin{equation}
			C_P - C_V = R
		\end{equation}

		\item Variation in internal energy:
		\begin{equation}
			\mathrm{d}E = C_V\mathrm{d}T
		\end{equation}

		\item Variation in enthalpy:
		\begin{equation}
			\mathrm{d}H = C_P\mathrm{d}T
		\end{equation}

		\item Variation in entropy:
		\begin{equation}
			\mathrm{d}S = \frac{C_P}{T}\mathrm{d}T - R\frac{\mathrm{d}P}{P}
		\end{equation}

	\end{enumerate}

	\subsection{NASA Polynomials}
	
	In its simplest form, the Ideal Gas Equation of State (IG EOS) assumes
	constant heat capacities, leading to linear relationships between enthalpy
	and internal energy with respect to temperature. However, this assumption is
	not valid across a wide range of temperatures. For example, in the field of
	combustion, the NASA polynomial model is commonly used to describe heat
	capacities more accurately. The NASA-7 model, for instance, expresses heat
	capacities in the following form [10]:

	\begin{equation}
		\frac{C_p}{R} = a_1 + a_2 + a_3 T^2 + a_4 T^3 + a_5 T^4
	\end{equation}

	\begin{equation}
		\frac{H}{RT} = a_1 + \frac{a_2}{2}T + \frac{a_3}{3}T^2 + \frac{a_4}{4}T^3 + \frac{a_5}{5}T^4 + \frac{a_6}{T}
	\end{equation}

	\begin{equation}
		\frac{S}{R} = a_1 \ln T + a_2 T + \frac{a_3}{2}T^2 + \frac{a_4}{3}T^3 + \frac{a_5}{4}T^4 + a_7
	\end{equation}
	The ideal gas equation of state (EOS) is a straightforward yet accurate
	model for describing gas behavior at high temperatures and relatively low
	pressures. However, it falls short in representing condensed fluids. Despite
	this limitation, the ideal gas model has been extensively studied and
	documented over the years. Consequently, many thermodynamic problems
	transition from the Real Gas (RG) state to the Ideal Gas (IG) state using
	residual or departure functions (see Section 2.4.3.2) to leverage the wealth
	of existing knowledge associated with the IG model.


	\subsection{Compressibility Factor}

	Real fluids at low density and high temperature can be accurately modeled by
	the perfect gas equation of state (EOS). However, at lower temperatures or
	higher densities, a real fluid deviates significantly from ideal gas
	behavior, especially during phase changes, such as when it condenses from a
	gas to a liquid or deposits from a gas to a solid. This deviation is
	quantified by the compressibility factor, $Z$, which indicates how much the
	fluid's behavior differs from that of an ideal gas:

	\begin{equation}
		Z = \frac{PV}{RT} =\frac{Pv}{rT} = \frac{P}{\rho RT} 
	\end{equation}

\section{Supercritical Fluids}

Supercritical phenomena are crucial in various industrial applications,
particularly in power cycles and energy recovery, which are the focus of this
work. This section is dedicated to understanding the key differences between
supercritical and subcritical fluids, with an emphasis on their role in power
generation.

Supercritical fluids, especially supercritical carbon dioxide ($CO_2$) and water,
are increasingly important in the development of advanced power cycles due to
their exceptional thermodynamic properties. These fluids are used to enhance
efficiency, reduce the size of equipment, and minimize environmental impact in
power generation systems. Here are some key applications:

    Supercritical $CO_2$ Power Cycles: Brayton Cycle: Supercritical $CO_2$ is a
        promising working fluid in closed-loop Brayton cycles, where it operates
        at temperatures and pressures above its critical point. These cycles
        have the potential to surpass the efficiency of traditional steam
        Rankine cycles, primarily because supercritical $CO_2$ has a higher
        density, which allows for more compact and efficient turbomachinery.
        This increased efficiency translates into lower fuel consumption and
        reduced greenhouse gas emissions, making supercritical $CO_2$ Brayton
        cycles a leading candidate for next-generation power plants. Waste Heat
        Recovery: One of the significant advantages of supercritical $CO_2$ cycles
        is their ability to recover waste heat from industrial processes, gas
        turbines, and even nuclear reactors. By utilizing waste heat, these
        systems can generate additional electricity without burning extra fuel,
        improving the overall efficiency of energy systems and reducing the
        carbon footprint of power generation.

    Supercritical Water Power Cycles: Supercritical Water-Cooled Reactors
        (SCWRs): In nuclear power generation, supercritical water is used as a
        coolant in advanced reactor designs. SCWRs operate at supercritical
        pressures and temperatures, allowing for higher thermal efficiency
        compared to conventional reactors. This not only improves the power
        output but also contributes to the safety and economic viability of
        nuclear energy.

    Supercritical Steam Cycles: Traditional fossil fuel power plants are
        evolving with the adoption of supercritical and ultra-supercritical
        steam cycles. By operating at temperatures and pressures above the
        critical point of water, these power plants achieve higher thermal
        efficiencies, reducing fuel consumption and emissions. The move towards
        supercritical steam cycles represents a significant advancement in
        making coal-fired power generation cleaner and more efficient.
		
    Hybrid Power Cycles: Integration with Renewable Energy: Supercritical $CO_2$
        and water cycles are being integrated with renewable energy sources,
        such as concentrated solar power (CSP) and geothermal energy. In CSP
        plants, supercritical $CO_2$ is used to transfer heat from the solar field
        to the power block, allowing for more efficient energy conversion.
        Similarly, in geothermal systems, supercritical fluids can enhance
        energy extraction from deep, high-temperature reservoirs, providing a
        reliable and sustainable source of electricity. Combined Cycle Plants:
        In combined cycle power plants, supercritical $CO_2$ can be used in
        conjunction with gas turbines to further increase overall efficiency.
        The exhaust heat from the gas turbine is used to generate additional
        power in a supercritical $CO_2$ cycle, making these plants among the most
        efficient forms of fossil fuel-based power generation.

    Environmental Benefits and Future Potential: Carbon Capture and Utilization:
        Supercritical $CO_2$ power cycles not only improve efficiency but also
        facilitate carbon capture and utilization (CCU). By integrating CCU with
        supercritical $CO_2$ cycles, power plants can reduce their carbon emissions
        while potentially producing valuable products like synthetic fuels. This
        dual benefit supports the transition to more sustainable energy systems.
        Future Research and Development: Ongoing research into supercritical
        fluid power cycles is focused on optimizing system design, improving
        materials that can withstand the harsh conditions of supercritical
        operation, and integrating these cycles with emerging technologies. The
        goal is to create highly efficient, low-emission power plants that can
        meet the growing global demand for electricity while minimizing
        environmental impact.

In conclusion, supercritical fluids are at the forefront of power generation
technology, offering significant advantages in efficiency, compactness, and
environmental sustainability. As research and development continue, these
systems are expected to play a central role in the future of energy, making them
a critical area of study for advancing power cycles and reducing the carbon
footprint of electricity generation.

In order to star with a better comprehension of the supercritical state
comprehension, some reduced parameters are introduced:

\begin{equation}
	Pr = \frac{P}{P_c}
\end{equation}

\begin{equation}
	T_r = \frac{T}{T_c}
\end{equation}

\begin{equation}
	V_r = \frac{V}{V_c}
\end{equation}

\begin{equation}
	\rho_r = \frac{\rho}{\rho_c}
\end{equation}

where subscript $c$ denotes the parameter's value at the critical point.

	\subsection{Physical Structure of Supercritical Fluids}

	The first topic to address is the nature of supercritical fluids. It is
	well-known that, in the supercritical state, there is no distinctly separate
	phase (see Figure 2.1) [14].

	Using molecular dynamics simulations of liquid and transcritical states at a
	reduced temperature of 0.5 and reduced pressures of 0.7, 1.4, and 3.0,
	Banuti [15] demonstrated that there is no need to separate quadrants IL and
	IV in Figure 2.1. This is because densely packed fluids in IL exhibit the
	same physical behavior as their supercritical counterparts in IV. For gases,
	however, the equivalency between the supercritical fluid and the ideal gas
	model in the ($P_r,T_r$) plane is reduced to a line. It is important to
	note that, up to $P_r=3$ and for $T_r>2$, describing a supercritical
	fluid as an ideal gas results in a compressibility factor error of less than
	5$\%$, as shown in Figure 2.2.

	Consequently, a pseudo-transition should exist, dividing the supercritical
	region, as illustrated in Figure 2.1. Recent independent studies by Gorelli
	et al. [17] and Simeoni et al. [18], using X-ray scattering, have shown that
	sound dispersion, typically observed only in liquids, indeed occurs in a
	supercritical transition. This suggests that there are two distinct
	pseudo-states within supercritical matter, implying the existence of a
	mechanism analogous to a subcritical phase transition.

	\subsection{Pseudo-Boiling} % entre [] pour le texte dans la TOC
		\subsubsection{The Widom line}
		The existence of this pseudo-transition is confirmed at loci where the
		isobaric heat capacity ($C_p$) exhibits a relative maximum, either at
		constant pressure or constant temperature, in the ($P_r,T_r$) plane.
		These two definitions correspond to the Widom Line and the Characteristic
		Isobaric Inflection Curve (CIIC), respectively. Recently, Lamorgese et al.
		compiled classical cubic Equation of State (EOS) results to map the Widom
		Line locus [19].

		Banuti, however, advanced this understanding further. By applying the
		theoretical definition of an equilibrium state—where Gibbs free energy is
		equal—and using a correlation from Waring [20] along with a
		Soave-Redlich-Kwong-based calculation of the subcritical coexistence line's
		slope, he proposed a similarity law that aligns both the coexistence and
		Widom lines across a range of fluids [21]. The strength of Banuti's approach
		lies in its solid theoretical foundation, its adaptability to various
		fluids, and its simplicity.

		\begin{equation}
			P_r = exp\left[\frac{A_s}{min(T_r,1)}(T_r-1)\right]
		\end{equation}

		where $A_s$ is the critical slope of the coexistence line and can be
		calculated using a cubic Equation of State. For example, applying
		the Soave-Redlich-Kwong EOS results in the following expression:

		\begin{equation}
			A_{SRK} = 5.51934 + 4.80640\omega - 0.537437\omega^2
		\end{equation}
		
		where $\omega$ is the acentric factor of the considered species [21]. The
		corresponding scaling parameter is defined as:

		\begin{equation}
			P_r^* = P_r^{\frac{A_0}{A_s}}
		\end{equation}

		where $A_0$ is the result of Eq. (2.19) for a zero acentric factor. This
		scaling parameter is illustrated graphically in Figure 2.3 below.

	\subsection{Supercritical Latent Heat}

	To highlight the differences between subcritical and supercritical phase
	transitions, Banuti [22] illustrated the contrast between low-pressure and
	high-pressure state transitions. As shown in Figure 2.4, at low pressures,
	the latent heat $\Delta h_{th}$ required to overcome intermolecular forces is
	gradually replaced by a thermal shielding effect $\Delta h_{th}$ at higher
	pressures. This effect forces the fluid to heat up before it can undergo the
	transition.

	Furthermore, the pseudo-boiling phenomenon involves both dynamic and
	thermodynamic state transitions [23]. Banuti et al. note in their study that
	at very high pressures, this transition loses its thermodynamic nature and
	is replaced by a dynamic shift from rigid to non-rigid liquids, which no
	longer display liquid-like dispersion behavior.

	Section 2.3 highlights the need for an accurate thermodynamic description of
	heat capacities. Models of transport properties, such as thermal conductivity,
	are discussed in Chapter 3 to ensure a proper understanding of supercritical
	(SC) behavior.

			
\section{Cubic EoS}

In this section, some of the most popular cubic equations of state (CEOS) are
introduced. Energy changes are expressed by applying the results from the
previous section. The widespread use of these equations is due to three key
reasons [4, 24]:

    They are applicable across a wide range of pressures and temperatures. They
    can represent compounds in both liquid and vapor phases. They are
    well-suited for the study of supercritical fluids.

All thermodynamic properties presented in this chapter are either obtained from
the NIST Chemistry WebBook [16] or calculated using some EOS.
		
	\subsection{A bit of history}
	This first section provides a brief historical overview of the EOS class
	under consideration.

	\subsubsection{The Corresponding States Principle and Criticalicity Condition}
	Van der Waals introduced the Corresponding States Principle (CSP) with the
	statement: 'Substances behave alike at the same reduced states. Substances
	at the same reduced states are in corresponding states.' This principle is
	of fundamental importance when considering supercritical fluids, as it
	introduces the critical point of a fluid (denoted by the subscript $\phi_c$) as
	a scaling factor, allowing for comparisons between different compounds. The
	concept gave rise to the widely used 'reduced' thermodynamic variables. The
	critical point serves as the ideal scaling reference for the application of
	the corresponding states principle due to the existence of specific
	criticality conditions, which are as follows:

	\begin{equation}
		\left(\frac{\partial P}{\partial V}\right)_{T=T_c}
	\end{equation}

	\begin{equation}
		\left(\frac{\partial^2 P}{\partial V^2}\right)_{T=T_c}
	\end{equation}

	These conditions express the fact that the saturation curve reaches a
	maximum at the critical point, and that the critical isotherm exhibits an
	inflection point at this same critical point. In fact, the attractive and
	repulsive parameters in cubic equations of state (CEOS) are derived from
	these critical conditions, making all compounds comparable under the
	Corresponding States Principle (CSP)

	\subsubsection{Van der Waals EOS}
	Historically, cubic equations of state (EOS) have been expressed in
	pressure-explicit form. They were specifically designed to account for
	intermolecular forces and accurately depict multiphase behavior. Van der
	Waals was the first to address this challenge, introducing the first cubic
	EOS, for which he was awarded the Nobel Prize in 1910 [25]. His
	straightforward formulation utilizes two constant parameters, aa and bb, to
	model attractive and repulsive forces, respectively

	\begin{equation}
		P = \frac{RT}{V-b} - \frac{a}{V^2}
	\end{equation}

	This formulation was developed into various equations of state (EOS)
	throughout the second half of the 20th century, particularly to refine the
	attractive term and make it more complex and accurate for different
	applications.

	\subsubsection{Redlich and Kwong's modification}
	
	In 1949, Redlich and Kwong, building on Van der Waals’ hard-sphere term,
	introduced an improved equation of state (EOS) by incorporating a more
	detailed temperature dependency into the attractive term [27]. This
	modification allowed their EOS to provide more accurate predictions of fluid
	behavior, particularly for gases near the critical point:

	\begin{equation}
		P = \frac{RT}{V-b} - \frac{a}{\sqrt(T_r)}\frac{1}{V(V+b)}
	\end{equation}

	Its widespread success stemmed from its adaptability, as Chueh and Prausnitz
	demonstrated that it could accurately predict both liquid and vapor
	saturation properties [28, 29].

	\subsubsection{Acentric factor introdction}

	Despite their success, both of these cubic EOS models fall short in
	representing the geometric complexity and polar properties of certain
	molecules. As illustrated in Figure 2.5, in complex polar
	compounds,intermolecular forces arise from locations other than the
	molecular centers.

	To address this issue, Pitzer introduced the acentric factor $\omega$ in 1955
	[30], as a measure of how much the thermodynamic properties of a substance
	deviate from those predicted by the Corresponding States Principle (CSP).

	\subsubsection{Soave's contribution}

	In 1972, Soave, recognizing the significance of Pitzer's acentric factor,
	extended the Redlich-Kwong equation of state (EOS) to create the widely
	known Soave-Redlich-Kwong (SRK) EOS [31]:

	\begin{equation}
		P = \frac{RT}{V-b}-\frac{a[1+K_{SRK}(\omega)(1-\sqrt(T_r))]^2}{V(V+b)}
	\end{equation}

	\begin{equation}
		K_{SRK}(\omega) = 0.480 + 1.574\omega -0.176\omega^2
	\end{equation}

	This formulation is particularly well-suited for accurately predicting vapor
	pressure data and behavior in the near-critical region [24].

	\subsubsection{Peng-Robinson equation of state}

	In 1976, Peng and Robinson (PR) recognized that the SRK EOS overestimated
	the critical compressibility factor of fluids ($Z_{c,SRK} = 1/3$). In response,
	they proposed an alternative form for the attractive term along with a more
	complex volume dependency [32]:

	\begin{equation}
		P = \frac{RT}{V-b}-\frac{a[1+K_{SRK}(\omega)(1-\sqrt(T_r))]^2}{V^2 +2V -b^2}
	\end{equation}

	\begin{equation}
		K_{SRK}(\omega) = 0.37464 + 1.54226\omega -0.26992\omega^2
	\end{equation}

	This formulation improved the accuracy of liquid volume predictions and
	provided more precise results around the critical point ($Z_{c,PR} = 0.307$).

	\subsection{Derived Properties for a Generic CEOS}
	This section focuses on establishing thermodynamic properties derived from
	Cubic Equations of State (CEOS). To provide a foundation for this, we begin
	with a few key reminders. Basic relationships between energy functions are
	introduced, followed by the application of the first law of thermodynamics
	to present fundamental equations that link differential expressions of
	various thermodynamic variables. Maxwell's relations are also recalled.

	Next, we demonstrate that Helmholtz free energy can serve as a generating
	function for other thermodynamic quantities. The concept of residual properties
	is then introduced, and these are applied to CEOS to determine the final
	expressions for important state functions, based on a generic CEOS.

	\subsubsection{Few reminders}

	Basic Definitions: In this section, we revisit the definitions of key
	thermodynamic quantities, including specific enthalpy $h$, Helmholtz free
	energy $A_W$, Gibbs free energy $g$, and the compressibility factor $Z$.

	\begin{equation}
		h = e + P/\rho
	\end{equation}

	\begin{equation}
		A_W = e + Ts
	\end{equation}

	\begin{equation}
		g = h - Ts
	\end{equation}

	\begin{equation}
		Z = \frac{P}{\rho RT}
	\end{equation}

	Fundamental equations: At this point, we need to establish relationships between the energetic
	variables and the PVT (Pressure-Volume-Temperature) state of a system. The
	first law of thermodynamics provides the fundamental equation for internal
	energy ee (see Eq. XXX). By combining this with the definitions of
	specific enthalpy, Helmholtz free energy, and Gibbs free energy, we derive
	the remaining fundamental equations. These equations are presented for a
	single species or a mixture of constant composition. Additionally, to
	transition between volume and density formulations, the following identity
	is applied: $dv = -\frac{d\rho}{\rho ^2}$

	\begin{equation}
		de = Tds - Pdv = Tds + \frac{P}{\rho ^2}d\rho
	\end{equation}

	\begin{equation}
		dh = Tds + vdP = Tds + \frac{1}{\rho}dP
	\end{equation}

	\begin{equation}
		dA_W = -Pdv -sdT = \frac{P}{\rho ^2}d\rho - sdT
	\end{equation}

	\begin{equation}
		dg = vdP - sdT = \frac{1}{\rho}dP -sdT
	\end{equation}

	Maxwell rules: Some mathematical tools required to derive the so-called
	Maxwell relations are presented in Section A.1. The Maxwell relations,
	derived from Equations (2.32a) through (2.32d), are then given as follows:

	\begin{equation}
		\left(\frac{\partial T}{\partial \rho}\right)_s = \frac{1}{\rho ^2}\left(\frac{\partial P}{\partial s}\right)_{\rho}
	\end{equation}

	\begin{equation}
		\left(\frac{\partial T}{\partial P}\right)_s = \left(\frac{\partial 1/\rho}{\partial s}\right)_P 
	\end{equation}

	\begin{equation}
		\frac{1}{\rho ^2}\left(\frac{\partial P}{\partial T}\right)_{\rho} = -\left(\frac{\partial s}{\partial \rho}\right)_T
	\end{equation}

	\begin{equation}
		\left(\frac{\partial 1/\rho}{\partial T}\right)_P = -\left(\frac{\partial s}{\partial P}\right)_T
	\end{equation}

	These relations will soon be used to demonstrate that Helmholtz free energy
	serves as a generating function for other important thermodynamic variables.