\addchap{Introduction}

The increasing global demand for energy, coupled with the urgent need to reduce greenhouse gas emissions, has driven the search for more efficient and sustainable energy solutions. 
Traditional energy generation methods, such as coal-fired power plants and natural gas turbines, contribute significantly to environmental pollution and are often limited by their 
thermodynamic efficiencies. In this context, supercritical fluids, particularly supercritical carbon dioxide (sCO2), have gained attention due to their unique properties and 
potential to revolutionize energy generation systems.
Supercritical CO2 is a state of carbon dioxide achieved when it is heated and pressurized above its critical point (31.1°C and 7.38 MPa). 
In this supercritical state, CO2 exhibits properties of both a liquid and a gas, resulting in a dense, compressible fluid with excellent thermal conductivity and diffusivity. 
These properties make sCO2 an ideal working fluid for various thermodynamic cycles, especially the Brayton cycle.


