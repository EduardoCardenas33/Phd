\addchap{Introduction}

The increasing global demand for energy, coupled with the urgent need to reduce greenhouse gas emissions, has driven the search for more efficient and sustainable energy solutions. 
Traditional energy generation methods, such as coal-fired power plants and natural gas turbines, contribute significantly to environmental pollution and are often limited by their 
thermodynamic efficiencies. In this context, supercritical fluids, particularly supercritical carbon dioxide (s$\ce{CO2}$), have gained attention due to their unique properties and 
potential to revolutionize energy generation systems.
Supercritical $\ce{CO2}$ is a state of carbon dioxide achieved when it is heated and pressurized above its critical point (31.1°C and 7.38 MPa). 
In this supercritical state, $\ce{CO2}$ exhibits properties of both a liquid and a gas, resulting in a dense, compressible fluid with excellent thermal conductivity and diffusivity. 
These properties make s$\ce{CO2}$ an ideal working fluid for various thermodynamic cycles, especially the Brayton cycle.

\section*{Importance of Supercritical $\ce{CO2}$ in Energy Generation}

Supercritical $\ce{CO2}$ offers several advantages over traditional working fluids like steam and air, which make it particularly attractive for use in power generation systems:

1) Higher Thermal Efficiency: The use of s$\ce{CO2}$ in Brayton cycles can achieve higher thermal efficiencies compared to conventional steam Rankine cycles. This is due to the reduced compression work and improved heat transfer characteristics of s$\ce{CO2}$, leading to more efficient energy conversion.

2) Compact System Design: The high density of s$\ce{CO2}$ allows for more compact and lighter components, such as compressors and heat exchangers. This can result in significant reductions in the size and cost of power generation systems, making them more suitable for a range of applications, from large-scale power plants to distributed energy systems.

3) Environmental Benefits: Utilizing s$\ce{CO2}$ in power cycles can lead to reduced fuel consumption and lower $\ce{CO2}$ emissions, contributing to the mitigation of climate change. Additionally, s$\ce{CO2}$ cycles can integrate more easily with renewable energy sources, enhancing their overall sustainability.

4) Operational Flexibility: s$\ce{CO2}$ cycles can operate efficiently over a wide range of temperatures and pressures, providing greater flexibility in integrating with various heat sources, including solar thermal, nuclear, and waste heat recovery systems.

\section*{Applications of Supercritical $\ce{CO2}$ in Brayton Cycles}
The Brayton cycle, traditionally used in gas turbines, involves the compression of a working fluid, heating at constant pressure, expansion to produce work, and cooling at constant pressure. When s$\ce{CO2}$ is used as the working fluid in a Brayton cycle, the cycle can operate at higher efficiencies due to the favorable thermophysical properties of s$\ce{CO2}$.

In recent years, there has been significant research and development focused on s$\ce{CO2}$ Brayton cycles for power generation. These cycles have shown promise in a variety of applications, including:

    Supercritical $\ce{CO2}$ Power Plants: Integration of s$\ce{CO2}$ cycles in power plants to enhance thermal efficiency and reduce emissions.
    Waste Heat Recovery: Utilization of s$\ce{CO2}$ cycles to recover and convert waste heat from industrial processes into electricity.
    Solar Thermal Power: Application of s$\ce{CO2}$ cycles in concentrated solar power (CSP) systems to improve efficiency and reduce the cost of solar power generation.
    Nuclear Power: Incorporation of s$\ce{CO2}$ Brayton cycles in advanced nuclear reactors to achieve higher thermal efficiencies and enhance safety.

\section*{Objectives of the Thesis}
This thesis aims to explore the modeling of supercritical $\ce{CO2}$ flow and heat transfer in an innovative thermal machine structure. By employing a cubic equation of state for thermodynamic
properties, the Lattice Boltzmann Method (LBM) for fluid dynamics, and the Immersed Boundary Method (IBM) for fluid-solid interactions, this research seeks to provide a comprehensive
understanding of the behavior of supercritical $\ce{CO2}$ in practical applications.

\section*{Significance of the study}
The use of supercritical $\ce{CO2}$ in energy generation systems, such as Brayton cycles, presents a significant opportunity to improve the efficiency and sustainability of power plants.
The higher thermal efficiency of s$\ce{CO2}$ cycles can lead to reduced fuel consumption and lower greenhouse gas emissions, aligning with global efforts to mitigate climate change. 
Additionally, the compactness and potential cost reductions associated with s$\ce{CO2}$ systems make them attractive for a wide range of applications, from large-scale power plants 
to smaller, distributed energy systems.

\section*{Structure of the Thesis}
This thesis is organized into several chapters, each addressing a key aspect of the research:

\begin{itemize}
    \item \textbf{Literature Review:} A review of the current state of knowledge on supercritical fluids, Brayton cycles, and the methodologies employed in this research.
    \item \textbf{Theory and Models:} Detailed presentation of the cubic equation of state for CO\textsubscript{2}, the Lattice Boltzmann Method, and the Immersed Boundary Method.
    \item \textbf{Simulation Methodology:} Description of the numerical implementation, including the algorithms, boundary conditions, and validation of models.
    \item \textbf{Case Studies and Results:} Presentation and analysis of simulation results for various case studies involving supercritical CO\textsubscript{2} flow and heat transfer.
    \item \textbf{Discussion:} Interpretation of the results, comparison with existing literature, and discussion of the limitations and implications of the findings.
    \item \textbf{Conclusion and Future Work:} Summary of the key contributions of the thesis and suggestions for future research directions.
\end{itemize}

By addressing these topics, this thesis aims to contribute to the understanding and practical application of supercritical CO\textsubscript{2} in energy generation, paving the way for more 
efficient and sustainable power systems.